\documentclass[a4paper]{extarticle}
\usepackage[utf8]{inputenc}
\usepackage[italian]{babel}
\selectlanguage{italian}
\usepackage[table]{xcolor}
\usepackage{xcolor}
\usepackage{circuitikz}
\usetikzlibrary{positioning, circuits.logic.US}
\usetikzlibrary{shapes.geometric, arrows}
\usetikzlibrary {shapes.gates.logic.US, shapes.gates.logic.IEC, calc}
\tikzset {branch/.style={fill, shape = circle, minimum size = 3pt, inner sep = 0pt}}
\usetikzlibrary{matrix,calc}
\usepackage{multirow}
\usepackage{float}
\usepackage{geometry}
\usepackage{tabularx}
\usepackage{pgf-pie}
\usepackage{tikz}
\usepackage{amsmath}
\usepackage{amssymb}
\usepackage{color, soul}
\usepackage{fancyhdr}
\usepackage{graphicx}
\usepackage{subfig}
\graphicspath{ {./img/} }
\newtheorem{theorem}{Teorema}[section]
\newtheorem{corollary}{Corollario}[theorem]
\newtheorem{lemma}[theorem]{Lemma}

% Specifiche
\geometry{
 a4paper,
 top=20mm,
 left=30mm,
 right=30mm,
 bottom=30mm
}

\nocite{}

\pagestyle{fancy}
\fancyhf{}
\fancyhead[LO]{\nouppercase{\leftmark}}
\fancyfoot[CE, CO]{\thepage}
\addtolength{\headheight}{1em}
\addtolength{\footskip}{-0.5em}

\newcommand{\quotes}[1]{``#1''}
\renewcommand\tabularxcolumn[1]{>{\vspace{\fill}}m{#1}<{\vspace{\fill}}}
\renewcommand\arraystretch{}
\newcolumntype{P}{>{\centering\arraybackslash}X}

\title{\textbf{Università di Trieste\\ \vspace{1em}
Laurea in ingegneria elettronica e informatica}}
\author{Enrico Piccin - Corso di Analisi matematica II - Prof. Franco Obersnel}
\date{Anno Accademico 2022/2023 - 3 Ottobre 2022}

\begin{document}

\vspace{-10mm}
\maketitle

\tableofcontents
\newpage
\noindent
\begin{center}
    3 Ottobre 2022
\end{center}

\vspace{1em}
\noindent
\section{Introduzione}
Considerando un foglio di carta, dividendolo in due metà esatte, si ottiene $\frac{1}{2}$ del profilo quadrato di partenza. Considerando una delle due metà, e suddividendola ancora in due, si ottiene $\frac{1}{4}$ del profilo quadrato di partenza.
Ripetendo questo procedimento, si otterranno le seguenti frazioni del profilo quadrato originario: $\frac{1}{8}, \frac{1}{16}, \frac{1}{32}, \frac{1}{64}, ...$. Sommando tutte le frazioni di profilo quadrato, alla fine si otterrà il profilo quadrato di partenza, ossia la frazione $1$.
Ecco quindi che, contrariamente a quanto voleva sostenere \textbf{Parmenide}, \textbf{Zenone} scoprì che
\[\boxed{\frac{1}{2}+\frac{1}{4}+\frac{1}{8}+\frac{1}{16}+\frac{1}{32}+\frac{1}{64}+...=1 \rightarrow \sum_{n=1}^{+\infty} \left(\frac{1}{2}\right)^n=1}\]
Ciò non risulta essere banale: una somma di \textbf{infinite quantità positive} produce una quantità finita. Quello che si è ottenuto è una \textbf{serie (numerica) geometrica di ragione $\frac{1}{2}$}.

\vspace{1em}
\section{Serie numerica}
Di seguito si espone la definizione di \textbf{serie numerica}:

% Tabella per le definizione di concetti, etc...
\vspace{1em}
\rowcolors{1}{black!5}{black!5}
\setlength{\tabcolsep}{14pt}
\renewcommand{\arraystretch}{2}
\noindent
\begin{tabularx}{\textwidth}{@{}|P|@{}}
    \hline
    {\textbf{SERIE NUMERICA}}\\
    \parbox{\linewidth}{Data una successione $(a_n)_n$ con valori nel campo complesso $a_n \in \mathbb{C}$. Si consideri una nuova successione $(s_n)_n$ definita \textbf{per ricorrenza} come segue
    \[s_{n+1}=s_n+a_{n+1} \hspace{1em} \text{ posto } \hspace{1em} s_0=a_0\]
    Ciò significa che
    \begin{itemize}
        \item $s_0 = a_0$
        \item $s_1 = a_0 + a_1$
        \item $s_2 = a_0 + a_1 + a_2$
        \item e via di seguito...
    \end{itemize}
    La serie $a_0+a_1+a_2+...$ è la \textbf{coppia ordinata} delle due successioni, come mostrato di seguito
    \[\left((a_n)_n, (s_n)_n\right)\]
    ove la successione $(a_n)_n$ prende il nome \textbf{successioni dei termini generali}, mentre la successione $(s_n)_n$ si chiama successione delle \textbf{ridotte} o delle \textbf{somme parziali} della serie. \vspace{3mm}}\\
    \hline
\end{tabularx}

\vspace{2em}
\noindent
\textbf{Esempio}: Posto $a_1=\frac{1}{2}$ e il termine generale $a_n=\left(\frac{1}{2}\right)^n$, la ridotta sarà
\[s_n=\frac{1}{2}+\frac{1}{4}+\frac{1}{8}+...+\frac{1}{2^n}\]
osservando bene di partire da $n=1$ e non da $0$.

\vspace{1em}
\noindent
\subsection{Convergenza, divergenza e indeterminatezza di una serie}
Data una serie, ossia data una coppia di successioni, è possibile ora andare a studiare il comportamento della successione delle ridotte.

\vspace{1em}
\noindent
\subsubsection{Convergenza di una serie}
Di seguito si espone la definizione di \textbf{convergenza di una serie}:

% Tabella per le definizione di concetti, etc...
\vspace{1em}
\rowcolors{1}{black!5}{black!5}
\setlength{\tabcolsep}{14pt}
\renewcommand{\arraystretch}{2}
\noindent
\begin{tabularx}{\textwidth}{@{}|P|@{}}
    \hline
    {\textbf{CONVERGENZA DI UNA SERIE}}\\
    \parbox{\linewidth}{Se la successione delle ridotte di una serie è convergente, si dice che la serie è convergente e il limite della successione delle ridotte prende il nome di \textbf{somma della serie}.\\
    In altre parole, se \textbf{esiste finito} il
    \[\lim_{n \to +\infty} s_n = s \in \mathbb{C}\]
    allora la serie si dice \textbf{convergente} e il limite $s$ si dice \textbf{somma della serie} e si scrive
    \[\sum_{n=0}^{+\infty} a_n = s\]
    \textbf{Attenzione}: Molto spesso si utilizza la notazione sopra esposta per indicare sia la serie stessa, sia la sua somma, per cui può essere fuorviante. Lo si può capire dal contesto: una serie potrebbe non essere convergente, e quindi non avere una somma.\vspace{3mm}}\\
    \hline
\end{tabularx}

\vspace{2em}
\noindent
\textbf{Esempio}: Se si considera $a_n=1, \forall n$, per cui
\[1+1+1+... = \sum_{n=0}^{n} 1\]
allora la somma parziale è $s_n=n+1$, ovvero una successione divergente a $+\infty$:
\[\lim_{n \to +\infty} s_n = +\infty\]
Ciò significa che la serie non converge, ma è \textbf{divergente}, per cui non ha nemmeno una somma.

\vspace{1em}
\noindent
\textbf{Osservazione}: Si osservi che la divergenza a $+\infty$ di una serie ha significato solamente quando i termini generali sono sul campo reale: se una serie ha termine generico nel campo complesso, non può essere divergente a $+\infty$, in quanto non esiste un limite infinito nel campo complesso (a meno che non si consideri il modulo).

\vspace{1em}
\noindent
\subsubsection{Divergenza di una serie}
Di seguito si espone la definizione di \textbf{divergenza di una serie}:

% Tabella per le definizione di concetti, etc...
\vspace{1em}
\rowcolors{1}{black!5}{black!5}
\setlength{\tabcolsep}{14pt}
\renewcommand{\arraystretch}{2}
\noindent
\begin{tabularx}{\textwidth}{@{}|P|@{}}
    \hline
    {\textbf{DIVERGENZA DI UNA SERIE}}\\
    \parbox{\linewidth}{Se la successione delle ridotte di una serie (a termine generale reale) è divergente, si dice che la serie è divergente; in questo caso, la serie non presenta una somma.\\
    In altre parole, se data $a_n \in \mathbb{R}, \forall n$, e posto
    \[\lim_{n \to +\infty} s_n = +\infty \text{ o } - \infty\]
    la serie si dice \textbf{divergente}.\vspace{3mm}}\\
    \hline
\end{tabularx}

\vspace{2em}
\noindent
\textbf{Esempio}: Se $a_n = a \in \mathbb{R}$ \textbf{costante}, allora la serie con termine generale $a_n$
\[a_0+a_1+a_2+...\]
è necessariamente 
\begin{itemize}
    \item divergente a $+\infty$ se $a > 0$
    \item divergente a $-\infty$ se $a < 0$
    \item convergente, con somma $0$, se $a = 0$
\end{itemize}
\textbf{Attenzione}: se $a \neq 0$, ma $a \in \mathbb{C} - \mathbb{R}$, si dice semplicemente che la serie \textbf{non converge} (non ha senso parlare di divergenza).


\vspace{1em}
\noindent
\subsubsection{Indeterminatezza di una serie}
Di seguito si espone la definizione di \textbf{serie indeterminata}:

% Tabella per le definizione di concetti, etc...
\vspace{1em}
\rowcolors{1}{black!5}{black!5}
\setlength{\tabcolsep}{14pt}
\renewcommand{\arraystretch}{2}
\noindent
\begin{tabularx}{\textwidth}{@{}|P|@{}}
    \hline
    {\textbf{SERIE INDETERMINATA}}\\
    \parbox{\linewidth}{Una serie si dice \textbf{indeterminata} se non converge e non diverge.\vspace{3mm}}\\
    \hline
\end{tabularx}

\vspace{2em}
\noindent
\textbf{Esempio 1}: Per quello che si è visto, una serie a termine generale costante, complesso e non reale, è indeterminata.

\vspace{1em}
\noindent
\textbf{Esempio 2}: Un esempio di serie a termini reali, ma indeterminata, è la \textbf{serie di Grandi}, definita così:
\[\sum_{n=0}^{+\infty} (-1)^n\]
per cui $s_0=(-1)^0=1$ e $s_1=a_0+a_1=1+(-1)^1=0$. Pertanto si ha che
\begin{itemize}
    \item $s_n=1$ se $n$ è pari
    \item $s_n=0$ se $n$ è dispari
\end{itemize}
Per cui si ha che
\[\lim_{n \to +\infty} s_0 = ? \text{ non esiste}\]
E per dimostrare che non esiste, si può semplicemente dimostrare che due sotto-successioni della successione delle somme parziali convergono a limiti diversi (ossia la sotto-successioni degli indici pari e quella dei dispari); infatti:
\begin{itemize}
    \item $\displaystyle{\lim_{k \to +\infty} s_{2k} = 1}$
    \item $\displaystyle{\lim_{k \to +\infty} s_{2k+1} = 0}$
\end{itemize}
per cui sono state ottenute due sotto-successioni che presentano limite differente: per il teorema dell'unicità del limite e il teorema del limite delle sotto-successioni di una successione, si conclude che la successione delle somme parziali è indeterminata.

\vspace{1em}
\noindent
\textbf{Osservazione}: La serie di Grandi è una serie che può essere usata per dimostrare l'esistenza di Dio, in quanto commutando fra di loro i differenti termini può essere fatta convergere a qualsiasi (o quasi) numero finito.\\
Se, infatti, si considerano le somme
\begin{itemize}
    \item $(1-1)+(1-1)+(1-1)+...=0$
    \item $1+(-1+1)+(-1+1)+...=1$
    \item $(1+1)+(-1+1)+(-1+1)=2$
\end{itemize}
si ottengono serie che convergono a qualunque valore (tranne uno). In generale, infatti, se una serie è indeterminata, si possono commutare gli addendi della stessa e ottenere la convergenza a qualunque numero.

\vspace{1em}
\noindent
\subsection{Serie geometrica}
Si è osservato che
\[\sum_{n=1}^{+\infty} \left(\frac{1}{2}\right)^n=1\]
per cui è ovvio che partendo con $n=0$, si ottiene
\[\sum_{n=0}^{+\infty} \left(\frac{1}{2}\right)^n=2\]
Più in generale, si fornisce di seguito la definizione di \textbf{serie geometrica}: 

% Tabella per le definizione di concetti, etc...
\vspace{1em}
\rowcolors{1}{black!5}{black!5}
\setlength{\tabcolsep}{14pt}
\renewcommand{\arraystretch}{2}
\noindent
\begin{tabularx}{\textwidth}{@{}|P|@{}}
    \hline
    {\textbf{SERIE GEOMETRICA}}\\
    \parbox{\linewidth}{Si dice \textbf{serie geometrica} di ragione $z \in \mathbb{C}$ la serie del tipo
    \[1+z+z^2+z^3+... \rightarrow \sum_{n=0}^{+\infty} z^n\]
    che, tuttavia, palesa un problema di fondo: se si sceglie $z=0$, naturalmente si incorre nell'ambiguità
    \[0^0 + 0^1 + ...\]
    ma $0^0$ è una scrittura che non ha significato. Tuttavia, in questo particolare caso, si considera $0^0=1$, in modo tale da essere coerenti con la scrittura $1+z+z^2+z^3+...$ impiegata in precedenza.\vspace{3mm}}\\
    \hline
\end{tabularx}

\vspace{1em}
\noindent
\textbf{Osservazione}: Data la serie seguente
\[\sum_{n=0}^{+\infty} z^n\]
per cui la ridotta è
\[s_n=1+z+z^2+...+z^n\]
che può anche essere riscritto come
\[s_n=1+z+z^2+...+z^n=1+z \cdot \left(1+z+...+z^{n-1}\right)\]
dove $1+z+...+z^{n-1}=s_{n-1}$. Da cui si evince che, sommando e sottraendo per la medesima quantità $z^n$, si ottiene
\[s_n = 1+z \cdot \left(\underbrace{1+z+...+z^{n-1}+z^n}_{s_n} - z^n\right)\]
che diviene, quindi:
\[s_n = 1 + z \cdot s_n - z^{n+1} \hspace{1em} \rightarrow \hspace{1em} s_n - z \cdot s_n = 1 - z^{n+1} \hspace{1em} \rightarrow \hspace{1em} s_n \cdot (1-z) = 1 - z^{n+1} \hspace{1em} \rightarrow \hspace{1em} s_n = \frac{1-z^{n+1}}{1-z}\]
posto $z \neq 1$ (ma il caso $z=1$ è facilmente risolubile, per quanto osservato nel caso di una serie a termine generale costante).\\
Di seguito si espone, quindi, il comportamento della serie geometrica a seconda della sua ragione $z$:

% Tabella per le definizione di concetti, etc...
\vspace{1em}
\rowcolors{1}{black!5}{black!5}
\setlength{\tabcolsep}{14pt}
\renewcommand{\arraystretch}{2}
\noindent
\begin{tabularx}{\textwidth}{@{}|P|@{}}
    \hline
    {\textbf{COMPORTAMENTO DELLA SERIE GEOMETRICA}}\\
    \parbox{\linewidth}{Per quanto osservato in precedenza, si ha che:
    \[\sum_{n=0}^{+\infty} z^n = \lim_{n \to +\infty} s_n = \lim_{n \to + \infty} \frac{1-z^{n+1}}{1-z}\]
    posto $z \neq 1$, che diviene
    \begin{itemize}
        \item $\displaystyle{\frac{1}{1-z}}$ se $\vert z \vert < 1$.
        \item non converge se $\vert z \vert > 1$, tuttavia, si può dire che
        \begin{itemize}
            \item se $z \in \mathbb{R}$ e $z \geq 1$, diverge a $+\infty$
            \item se $z \in \mathbb{C}$ e $\vert z \vert \geq 1$ (ovvero può essere anche un numero negativo), posto $z \notin \left]1,+\infty \right[$ (ossia diverso dal caso precedente), nel caso di $n$ pari si sommano quantità positive, nel caso di $n$ dispari si sommano quantità negative, per cui la serie oscilla e quindi è indeterminata.
        \end{itemize}
    \end{itemize}
    \vspace{1mm}}\\
    \hline
\end{tabularx}



\vspace{1em}
\noindent
\textbf{Osservazione}: Si osservi che la serie geometrica è l'unica per cui si riesce a calcolare la somma, in quanto è l'unica di cui è possibile esprimere la ridotta in modo generale.\\
Altrimenti, gestire le ridotte diviene molto complesso.

\vspace{1em}
\noindent
\textbf{Esempio}: Si consideri la seguente serie
\[\sum_{n=2}^{+\infty} \cos^{n}(1)\]
che è una serie geometrica di ragione $\cos(1)$, ove $\left \vert \cos(1) \right \vert < 1$, per cui converge. La somma di tale serie, quindi, è facilmente determinabile secondo quanto visto in precedenza, tenendo conto che $n$ parte da 2, per cui bisogna sottrarre $\cos^0(1)=1$ e $\cos^1(1)=\cos(1)$. Da ciò si evince che la serie converge a 
\[\frac{1}{1 - \cos(1)} - 1 - \cos(1) = \frac{1 - 1 + \cos(1) - \cos(1) + \cos^2(1)}{1 - \cos(1)} = \frac{\cos^2(1)}{1 - \cos(1)}\]

\vspace{1em}
\noindent
\textbf{Osservazione}: La somma della serie geometrica di ragione $z \in \mathbb{C}$ è indeterminata se $\vert z \vert > 1$, per quanto già visto.\\
Inoltre si ha che la serie
\[\sum_{n=1}^{+\infty} \left(\frac{2i + x}{4}\right)^n\]
è convergente se
\[\left \vert \frac{2i + x}{4}\right \vert < 1\]
ma ricordando come si calcola il modulo di un numero complesso si ottiene
\[\left \vert 2i + x \right \vert = \sqrt{4+x^2}\]
e quindi 
\[\sqrt{4+x^2} < 4 \hspace{1em} \rightarrow \hspace{1em} 4 + x^2 < 16 \hspace{1em} \rightarrow \hspace{1em} x^2 < 12 \hspace{1em} \rightarrow \hspace{1em} \vert x \vert < \sqrt{12} \hspace{1em} \rightarrow \hspace{1em} \vert x \vert < 2 \sqrt{3}\]
E poi, ovviamente, la serie di Grandi è il tipico esempio di serie indeterminata, per cui la sua somma non può essere definita.

\newpage
\noindent
\begin{center}
    5 Ottobre 2022
\end{center}
Una serie è costituita da $2$ successioni: la successione dei termini generali e la successione delle ridotte o somme parziali: quando si opera con le serie, risulta fondamentale distinguere le due successioni.\\
Una tra le serie più note è la serie geometrica, di ragione $z \in \mathbb{C}$, la quale converge se il modulo della ragione è minore di $1$. Non converge in caso contrario, ma in particolare
\begin{itemize}
    \item se la ragione $z$ è un numero reale, $z \in \mathbb{R}$, e $z \geq 1$, allora la serie diverge a $+\infty$;
    \item se la ragione $z$ è un numero complesso, con $\vert z \vert \geq 1$ e $z \notin ]1,+\infty[$, allora la serie è indeterminata.
\end{itemize} 
In generale, non si può parlare di divergenza a $+\infty$ o $-\infty$ in campo complesso, in quanto in esso è \textbf{assente la relazione d'ordine} e quindi non esiste un limite infinito.

\vspace{1em}
\noindent
\textbf{Esempio}: Si consideri l'esempio seguente:
\[\sum_{n=0}^{+\infty} \frac{\cos(n)}{2^n}\]
Tale serie presenta come termine generale
\[a_n = \frac{\cos(n)}{2^n}\]
ma è vero che $-1 \leq \cos(n) \leq 1$, per cui
\[-\frac{1}{2^n} \leq a_n \leq \frac{1}{2^n}\]
Per dimostrare che anche la serie in esame converge, è sufficiente considerare $s_n^-$ e $s_n^+$, rispettivamente la ridotta $n$-esima della serie geometrica di ragione $-\frac{1}{2}$ e $\frac{1}{2}$, come segue
\[s_n^- = -1 - \frac{1}{2} - ... - \frac{1}{2^n} \hspace{1em} \text{e} \hspace{1em} s_n^+ = 1 + \frac{1}{2} + ... + \frac{1}{2^n}\]
per cui
\[s_n^- \leq s_n \leq s_n^+\]
e per il \textbf{teorema del confronto esiste finito} il seguente limite
\[\lim_{n \to +\infty} s_n \in \mathbb{R}\]
e quindi la serie
\[\sum_{n=0}^{+\infty} \frac{\cos(n)}{2^n}\]
converge.

\vspace{1em}
\noindent
\subsection{Teorema del confronto per le serie}
Di seguito si espone il fondamentale \textbf{teorema del confronto per le serie}:

\vspace{1em}
\noindent
\begin{theorem} \textbf{Teorema del confronto per le serie}\\
    Siano $a_n,b_n,c_n \in \mathbb{R}$ tali che $a_n \leq b_n \leq c_n, \forall n$ (anche se sarebbe sufficiente richiedere che ciò sia vero \textbf{definitivamente}, ossia $\exists n_0 \in \mathbb{N}$ tale che la disuguaglianza di cui sopra è valida $\forall n \geq n_0$). Siano convergenti le serie
    \[\sum a_n \hspace{1em} \text{e} \hspace{1em} \sum c_n\]
    allora è convergente anche la serie
    \[\sum b_n\]
    ed è tale la stima della somma della serie:
    \[\sum a_n \leq \sum b_n \leq c_n\]
    che è una stima valida $\forall n$, oppure $\forall n \geq n_0$ (a seconda che sia stato richiesto $\forall n$, oppure definitivamente).
\end{theorem}

\vspace{1em}
\noindent
\textbf{Osservazione}: Si osservi il caso particolare per cui $a_n=0, \forall n$ (ossia serie a termini positivi, cioè una serie per cui tutti i termini della successione dei termini generale sono positivi) oppure quelle per cui $c_n=0, \forall n$ (ossia serie a termini negativi, vale a dire serie per cui tutti i termini della successione dei termini generali sono negativi).\\
In questi casi, infatti, è sufficiente considerare un limitazione superiore (o inferiore, rispettivamente) per concludere la convergenza.\\

\vspace{1em}
\noindent
\textbf{Dimostrazione - IMPORTANTE}: Si dimostri che il carattere di una serie non dipende da quello che accade su un numero finito di termini.

\vspace{1em}
\noindent
\textbf{Esempio}: Si consideri la serie
\[\sum_{n=0}^{+\infty} \frac{1}{2^n} e^{100-n^2}\]
Essendo essa a termini positivi e maggiorata da
\[\sum_{n=0}^{+\infty} \frac{1}{2^n} e^{100}\]
per il teorema del confronto.

\vspace{1em}
\noindent
\textbf{Osservazione}: Si osservi che il termine generale $a_n < \frac{1}{2^n}$ quando ... continua ...

\vspace{1em}
\noindent
\textbf{Esempio}: Si consideri la serie seguente:
\[\sum_{n=1}^{+\infty} \cos \left(\frac{1}{n}\right) = + \infty\]
in quanto
\[\lim_{n \to +\infty} \cos \left(\frac{1}{n}\right)=1\]
ossia, per $n$ molto grande, nella serie si somma sempre $1$, per cui diverge.

\vspace{1em}
\noindent
\begin{theorem} \textbf{Condizione necessaria per la convergenza}\\
    Sia $\sum_{n=0}^{+\infty} a_n$ una serie convergente, allora
    \[\lim_{n \to \infty} a_n=0\]
\end{theorem}

\vspace{1em}
\noindent
\textbf{Dimostrazione}: Sia $s_{n+1} = s_n + a_{n+1}$, tale per cui
\[a_{n+1} = s_{n+1} - s_n\]
Siccome la serie è convergente per ipotesi ($s_{n+1}$ e $s_n$ convergono allo stesso limite):
\[\lim_{n \to +\infty} a_{n+1} = s_{n+1} - s_n = 0\]

\vspace{1em}
\noindent
\textbf{Osservazione}: Si osservi che esistono delle serie
\[\sum a_n\]
non convergenti, dove il
\[\lim_{n \to +\infty} a_n = 0\]
per questo si parla di condizione necessaria, e non sufficiente. Infatti è importante definire con quale velocità il termine generale vada a $0$: se è troppo lenta, nonostante sia infinitesima, la serie associata convergerà.

\vspace{1em}
\noindent
\subsection{Serie armonica}
Si consideri la serie seguente
\[\sum_{n=1}^{+\infty} \frac{1}{n}\]
che prende il nome di \textbf{serie armonica}. Per studiarne il comportamento, è sufficiente capire che \textbf{ogni serie può essere considerata come un integrale generalizzato}. Infatti, per definizione di integrale generalizzato:
\[\int_a^{+\infty} f(x) dx = \lim_{b \to +\infty} \int_a^b f(x) dx\]
allora se si considera la serie $a_1+a_2+a_3$, si definisce una funzione
\[f : [1,+\infty[ \longmapsto \mathbb{R}\]
nel modo seguente: essendo una successione una funzione (definita sui numeri naturali), si possono interpolare i valori di una successione tramite delle costanti, come nel seguito:
\[f(x)=a_n \hspace{1em} \text{ se } \hspace{1em} x \in [n,n+1[, \hspace{1em} \forall n \geq 1\]

\vspace{1em}
\noindent
\textbf{Osservazione}: Naturalmente si ha che
\[\int_{n}^{n+1} f(x) \cdot dx = a_n \cdot (n+1-n) = a_n\]
per cui è ovvio che
\[s_n=a_1+a_2+...+a_n=\int_1^{n+1} f(x) \cdot dx\]
Se $f$ è integrabile, allora
\[\int_1^{+\infty} f(x) \cdot dx = \lim_{b \to +\infty} \int_1^b f(x) \cdot dx\]
per cui, per il teorema del limite delle successioni, ogni successione che tende a $+\infty$ avrà lo stesso limite della funzione $f$, ossia
\[\lim_{n \to +\infty} s_n = \lim_{n \to +\infty} f(x) ...continua...\]

\vspace{1em}
\noindent
\textbf{Osservazione}: Si osservi che se la serie converge, per cui
\[\lim_{n \to +\infty} \int_1^{n+1} f(x) \cdot dx = \lim_{n \to +\infty} = s\]
è anche vero che $f$ è integrabile, in quanto, posto $[b]=n$, essendo $b < n+1$,
\[\int_1^b f(x) \cdot dx = \int_1^n f(x) cdot dx + \int_n^b f(x) dx\]
Allora, giacché
\[\int_n^b f(x) \cdot dx = a_n \cdot (b-n) \leq a_n\]
in quanto $b<n+1$ e $[b]=n$. Ma siccome la serie converge, allora il limite del termine generale è $0$, quindi
\[\int_1^b f(x) \cdot dx = \int_1^n f(x) cdot dx\]
come esposto da teorema seguente:

\vspace{1em}
\noindent
\begin{theorem}
    Sia $a_1+a_2+...$ una serie e sia $f$ la funzione precedentemente descritta, allora $f$ è integrabile in senso generalizzato su $[1,+\infty[$ se e solo se la serie converge... continua...
\end{theorem}

\vspace{1em}
\noindent
\textbf{Osservazione}: Se si considera la funzione
\[g(x)=\frac{1}{x}\]
allora, sapendo che
\[g(x) \leq f(x), \forall x \in [1,+\infty[\]
in quanto $f$ è la funzione a tratti precedentemente definita. Per cui, siccome $g(x)$ non è integrabile, non lo è nemmeno la $f$ (per il teorema del confronto degli integrali generalizzati), pertanto la serie
\[\sum_{n=1}^{+\infty} \frac{1}{n}\]
per il teorema precedentemente esposto, non converge.

\vspace{1em}
\noindent
\subsubsection{Serie armonica generalizzata}
È noto che la serie armonica non converge. Non sorprende, però, sapere che tale serie è divergente a $+\infty$, ovvero
\[\sum{n=1}^{+\infty} \frac{1}{n} = + \infty\]
Pertanto, se si considera
\[\sum{n=1}^{+\infty} \frac{1}{\sqrt{n}}\]
essa è necessariamente
\[\frac{1}{\sqrt{n}} \geq \frac{1}{n}, \hspace{1em} \forall n \geq 1\]
per cui, per il teorema del confronto, diverge a $+\infty$. Ciò risulta vero per ogni
\[\frac{1}{n^\alpha} \geq \frac{1}{n}, \forall n \geq 1 \hspace{1em} \text{ se } 0 < \alpha \leq 1\]
In generale, tuttavia, sappiamo
\[\int_1^{+\infty} \frac{1}{x^\alpha} \cdot dx = \left[\frac{1}{-\alpha+1} \cdot x^{-\alpha+1}\right]_1^{+\infty} = \frac{1}{\alpha-1}\]
Tuttavia, impiegando la funzione definita in precedenza (da $n$ a $n+1$), siccome sarà maggiore di $g(x)=\frac{1}{x^2}$, non è possibile stabilire se essa sia integrabile o meno.\\
Per tale ragione si definisce
\[h(x)=a_n \hspace{1em} \text{ se } \hspace{1em} x \in ]n-1,n]\]
allora
\[\int_{n-1}^n h(x) \cdot dx = a_n\]
Da ciò segue che
\[\int_1^{+\infty}=a_2+a_3+...+=\sum_{n=2}^{+\infty} a_n\]
Pertanto, siccome
\[\sum_{n=2}^{+\infty} \frac{1}{n^2} = \int_1^{+\infty} h(x) \cdot dx \leq \int_1^{+\infty} \frac{1}{x^2} \cdot dx = 1\]
ciò permette di concludere che la serie armonica generalizzata
\[\sum_{n=1}^{+\infty} \frac{1}{n^\alpha} \frac{1}{n^\alpha}\]
con $\alpha>0$ è divergente a $+\infty$ se $\alpha \in ]0,1]$, è convergente se $\alpha>1$ con somma
\[s \leq 1 + \frac{1}{\alpha - 1} = \frac{\alpha}{\alpha-1}\]
dal momento che l'integrale
\[\int_1^{+\infty} h(x) \cdot dx = \sum_{n=2}^{+\infty} \frac{1}{n^\alpha} \hspace{1em} \text{ ovvero } \hspace{1em} \sum_{n=1}^{+\infty}  \cdot dx = 1 +\]

\vspace{1em}
\noindent
\textbf{Esercizio 1}: Si consideri la serie
\[\sum_{n=2}^{+\infty} \frac{1}{\log(n)}\]
che, ovviamente, diverge in quanto
\[\frac{1}{\log(n)} \geq \frac{1}{n}\]
e siccome $\frac{1}{n}$ diverge, per il teorema del confronto, diverge anche $\frac{1}{\log(n)}$.

\vspace{1em}
\noindent
\textbf{Esercizio 2}: Si consideri la serie
\[\sum_{n=1}^{+\infty} \frac{1}{n \cdot (\log(n))^\alpha}\]
Per capire se essa diverga o meno, si considera l'integrale
\[\]
... continua ...

\vspace{1em}
\noindent
\subsection{Serie a termini (reali) positivi}
Si consideri $a_n \geq 0, \forall n$ (anche se sarebbe sufficiente \textbf{definitivamente}, ossia da un certo $n$ in poi), per il \textbf{teorema dell'Aut-Aut}, o converge, o diverge.\\
Ciò spiega perché la serie armonica diverga a $+\infty$, in quanto si è dimostrato che non converge.\\
Il teorema dell'Aut-Aut si aggiunge al teorema del confronto

\newpage
\noindent
\begin{center}
    7 Ottobre 2022
\end{center}
\noindent
Dopo aver analizzato la condizione necessaria per la convergenza, è stato anche considerato il fatto che una serie può essere sempre considerata come un integrale generalizzato. Un esempio fondamentale di serie di confronto è anche la serie armonica.\\
Di seguito si espongono alcuni teoremi fondamentali per la convergenza/divergenza di una serie.

\vspace{1em}
\noindent
\subsection{Teorema dell'Aut-Aut per le serie a termini (reali) positivi}
Si supponga che la serie
\[a_1+a_2+\dots+a_n+\dots=\sum_{n=1}^{+\infty} a_n\]
abbia termini positivi ($a_n > 0)$ o al più non negativi ($a_n \geq 0$). Allora essa converge o diverge. In altre parole, una serie a termini non negativi non può essere indeterminata.

\vspace{2em}
\noindent
\normalfont \normalsize
\textsc{Dimostrazione}: Supposto $a_n \geq 0, \forall n$ (anche se sarebbe sufficiente richiedere definitivamente), la successione delle ridotte è \textbf{crescente (anche in senso debole)}, tale per cui
\[s_{n+1}=s_n+a_{n+1} \geq s_n\]
Per il \textbf{teorema di esistenza del limite delle successioni monotone}, la successione delle ridotte ammette limite, ed esso è
\[\lim_{n \to +\infty} s_n = \text{ sup } \{s_n : n \in \mathbb{N}^+\}\]
Pertanto
\begin{itemize}
    \item se la successione delle ridotte è superiormente limitata, ovvero $\text{sup } \{s_n\} \in \mathbb{R}$, la serie è ovviamente convergente.
    \item se la successione delle ridotte è superiormente illimitata, per cui $\text{sup } \{s_n\} = +\infty$, la serie diverge a $+\infty$.
\end{itemize}
In ogni caso, però, \textbf{la serie non può essere indeterminata}.

\vspace{1em}
\noindent
\textbf{Osservazione}: Naturalmente la stessa cosa vale anche per successioni a termini negativi. L'importante è che sia verificata la condizione $a_n \geq 0$ oppure $a_n \leq 0$ infinitesimo.

\vspace{1em}
\noindent
\subsection{Criterio dell'ordine di infinitesimo per le serie a termini positivi}
Il teorema dell'Aut-Aut permette di dimostrare anche un altro importante criterio:

\begin{theorem}\textbf{Criterio dell'ordine di infinitesimo per le serie a termini positivi}\\
    Sia
    \[\sum_{n=0}^{+\infty} a_n\]
    una serie a termini positivi con termine generale infinitesimo
    \[\lim_{n \to +\infty} a_n = 0\]
    allora
    \begin{itemize}
        \item se esiste $\alpha > 1 \vert \text{ ord } a_n \geq \alpha$, la serie converge
        \item se esiste $\alpha > 1 \vert \text{ ord } a_n \leq 1$, la serie diverge
    \end{itemize}
\end{theorem}

\vspace{2em}
\noindent
\normalfont \normalsize
\textsc{Dimostrazione}: Supposto che la successione $a_n$ abbia come ordine di infinitesimo $\alpha$, con $\alpha > 1$, ossia
\[\lim \left \vert \frac{a_n}{\frac{1}{n^\alpha}}\right \vert = l \hspace{1em} \text{ posto } \hspace{1em} l \neq 0\]
allora, per definizione stessa di limite,
\[\forall \epsilon > 0, \exists n_\epsilon \in \mathbb{N} \vert \forall n > n_\epsilon \text{ si ha} n^\alpha < l + \epsilon\]
Per comodità, si sceglie $\epsilon = 1$, da cui
\[n^\alpha a_n < l+1\]
Ciò consente di affermare che $\forall n > n_\epsilon$ si ha che
\[0 \leq a_n \leq (l+1) \cdot \frac{1}{n^\alpha}\]
In questo modo si sta confrontando il termine generale $a_n$ con il termine generale della serie armonica generalizzata. Per il teorema del confronto, siccome definitivamente
\[a_n \leq (l+1) \cdot \frac{1}{n^\alpha}\]
e la serie armonica converge, in quanto $\alpha > 1$ ... continua ...

\vspace{1em}
\noindent
Supposto, ora, $\text{ord } a_n \leq 1$, si dimostri che la serie
\[\sum_{n=1}^{+\infty}a_n\]
diverge.\\
Il fatto che $\text{ord } a_n \leq 1$, significa che
\[\lim_{n \to +\infty} \frac{a_n}{\frac{1}{n}} = l\]
per cui se $l \in \mathbb{R} - \{0\}$ significa che $\text{ord } a_n = 1$, se $l = +\infty$, $\text{ord } a_n < 1$.\\
Nell'ipotesi che $l \in \mathbb{R} - \{0\}$, ovvero
\[\lim_{n \to +\infty} n \cdot a_n = l \in \mathbb{R} - \{0\}\] allora, per la definizione stessa di limite
\[\forall \epsilon > 0, \exists n_\epsilon \in \mathbb{N} \vert \forall n > n_\epsilon : \left \vert a_n-l \right \vert < \epsilon\]
Scelto, per comodità, $\epsilon=\frac{l}{2}$, e quindi ... continua ...
\[\]

\vspace{1em}
\noindent
\textbf{Osservazione}: In particolare, se $\exists \alpha \in \mathbb{R}, \alpha>1$, e si ha
\begin{itemize}
    \item $\text{ord } a_n > \alpha$, la serie converge
    \item $\text{ord } a_n \leq 1$, la serie diverge
\end{itemize}
Tuttavia, sapere che $\text{ord } a_n > 1$ non fornisce informazioni

\vspace{1em}
\noindent
\textbf{Esercizio 1}: La serie
\[\sum \frac{5n + \cos(n)}{3 + 2n^3}\]
è ovviamente convergente, in quanto $\text{ord } a_n = 2 > 1$.

\vspace{1em}
\noindent
\textbf{Esercizio 2}: La serie
\[\sum \frac{2\sqrt{n}}{n^2+n+1}\]
è ovviamente convergente, in quanto $\text{ord } a_n = \frac{3}{2} > 1$.

\vspace{1em}
\noindent
\textbf{Esercizio 3}: La serie
\[\sum \log \left(1-\frac{1}{n}\right)\]
non converge. La serie è a termini negativi, tuttavia si può fare
\[-\lim_{n \to +\infty} -\frac{\log \left(1-\frac{1}{n}}{\frac{1}{n}}=1\]
per cui $\text{ord } a_n = 1$.

\vspace{1em}
\noindent
\textbf{Esercizio 4}: La serie
\[\sum 1 - \cos \left(\frac{1}{n}\right)\]
è ovviamente convergente, in quanto $\text{ord } a_n = 2 > 1$.

\vspace{1em}
\noindent
\textbf{Esercizio 5}: La serie
\[\sum \frac{2^n}{(\log(n))^n} = \sum \left(\frac{2}{\log(n)}\right)^n\]
è ovviamente convergente, in quanto
\[\frac{2}{\log(n)} < \frac{2}{3} \rightarrow \log(n) > 3\]
per $n > e^3$, ma l'importante è che accada definitivamente, per cui la serie converge per confronto con la serie geometrica.

\vspace{1em}
\noindent
\textbf{Esercizio 6}: La serie
\[\sum \frac{2^n}{(\log(n))^n} = \sum \left(\frac{2}{\log(n)}\right)^n\]




\end{document}