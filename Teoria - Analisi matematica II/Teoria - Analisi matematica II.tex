\documentclass[a4paper]{extarticle}
\usepackage[utf8]{inputenc}
\usepackage[italian]{babel}
\selectlanguage{italian}
\usepackage[table]{xcolor}
\usepackage{xcolor}
\usepackage{circuitikz}
\usetikzlibrary{positioning, circuits.logic.US}
\usetikzlibrary{shapes.geometric, arrows}
\usetikzlibrary {shapes.gates.logic.US, shapes.gates.logic.IEC, calc}
\tikzset {branch/.style={fill, shape = circle, minimum size = 3pt, inner sep = 0pt}}
\usetikzlibrary{matrix,calc}
\usepackage{multirow}
\usepackage{float}
\usepackage{geometry}
\usepackage{tabularx}
\usepackage{pgf-pie}
\usepackage{tikz}
\usepackage{amsmath}
\usepackage{amssymb}
\usepackage{color, soul}
\usepackage{fancyhdr}
\usepackage{graphicx}
\usepackage{subfig}
\graphicspath{ {./img/} }
\newtheorem{theorem}{Teorema}[section]
\newtheorem{corollary}{Corollario}[theorem]
\newtheorem{lemma}[theorem]{Lemma}

% Specifiche
\geometry{
 a4paper,
 top=20mm,
 left=30mm,
 right=30mm,
 bottom=30mm
}

\nocite{}

\pagestyle{fancy}
\fancyhf{}
\fancyhead[LO]{\nouppercase{\leftmark}}
\fancyfoot[CE, CO]{\thepage}
\addtolength{\headheight}{1em}
\addtolength{\footskip}{-0.5em}

\newcommand{\quotes}[1]{``#1''}
\renewcommand\tabularxcolumn[1]{>{\vspace{\fill}}m{#1}<{\vspace{\fill}}}
\renewcommand\arraystretch{}
\newcolumntype{P}{>{\centering\arraybackslash}X}
\newcommand*\dif{\mathop{}\!\mathrm{d}}

\title{\textbf{Università di Trieste\\ \vspace{1em}
Laurea in ingegneria elettronica e informatica}}
\author{Enrico Piccin - Corso di Analisi matematica II - Prof. Franco Obersnel}
\date{Anno Accademico 2022/2023 - 3 Ottobre 2022}

\begin{document}

\vspace{-10mm}
\maketitle

\tableofcontents
\newpage
\noindent
\begin{center}
    3 Ottobre 2022
\end{center}

\vspace{1em}
\noindent
\section{Introduzione}
Considerando un foglio di carta, dividendolo in due metà esatte, si ottiene $\frac{1}{2}$ del profilo quadrato di partenza. Considerando una delle due metà, e suddividendola ancora in due, si ottiene $\frac{1}{4}$ del profilo quadrato di partenza.
Ripetendo questo procedimento, si otterranno le seguenti frazioni del profilo quadrato originario: $\frac{1}{8}, \frac{1}{16}, \frac{1}{32}, \frac{1}{64}, ...$. Sommando tutte le frazioni di profilo quadrato, alla fine si otterrà il profilo quadrato di partenza, ossia la frazione $1$.
Ecco quindi che, contrariamente a quanto voleva sostenere \textbf{Parmenide}, \textbf{Zenone} scoprì che
\[\boxed{\frac{1}{2}+\frac{1}{4}+\frac{1}{8}+\frac{1}{16}+\frac{1}{32}+\frac{1}{64}+...=1 \rightarrow \sum_{n=1}^{+\infty} \left(\frac{1}{2}\right)^n=1}\]
Ciò non risulta essere banale: una somma di \textbf{infinite quantità positive} produce una quantità finita. Quello che si è ottenuto è una \textbf{serie (numerica) geometrica di ragione $\frac{1}{2}$}.

\vspace{1em}
\section{Serie numerica}
Di seguito si espone la definizione di \textbf{serie numerica}:

% Tabella per le definizione di concetti, etc...
\vspace{1em}
\rowcolors{1}{black!5}{black!5}
\setlength{\tabcolsep}{14pt}
\renewcommand{\arraystretch}{2}
\noindent
\begin{tabularx}{\textwidth}{@{}|P|@{}}
    \hline
    {\textbf{SERIE NUMERICA}}\\
    \parbox{\linewidth}{Data una successione $(a_n)_n$ con valori nel campo complesso $a_n \in \mathbb{C}$. Si consideri una nuova successione $(s_n)_n$ definita \textbf{per ricorrenza} come segue
    \[s_{n+1}=s_n+a_{n+1} \hspace{1em} \text{ posto } \hspace{1em} s_0=a_0\]
    Ciò significa che
    \begin{itemize}
        \item $s_0 = a_0$
        \item $s_1 = a_0 + a_1$
        \item $s_2 = a_0 + a_1 + a_2$
        \item e via di seguito...
    \end{itemize}
    La serie $a_0+a_1+a_2+...$ è la \textbf{coppia ordinata} delle due successioni, come mostrato di seguito
    \[\left((a_n)_n, (s_n)_n\right)\]
    ove la successione $(a_n)_n$ prende il nome \textbf{successioni dei termini generali}, mentre la successione $(s_n)_n$ si chiama successione delle \textbf{ridotte} o delle \textbf{somme parziali} della serie. \vspace{3mm}}\\
    \hline
\end{tabularx}

\vspace{2em}
\noindent
\textbf{Esempio}: Posto $a_1=\frac{1}{2}$ e il termine generale $a_n=\left(\frac{1}{2}\right)^n$, la ridotta sarà
\[s_n=\frac{1}{2}+\frac{1}{4}+\frac{1}{8}+...+\frac{1}{2^n}\]
osservando bene di partire da $n=1$ e non da $0$.

\vspace{1em}
\noindent
\subsection{Convergenza, divergenza e indeterminatezza di una serie}
Data una serie, ossia data una coppia di successioni, è possibile ora andare a studiare il comportamento della successione delle ridotte.

\vspace{1em}
\noindent
\subsubsection{Convergenza di una serie}
Di seguito si espone la definizione di \textbf{convergenza di una serie}:

% Tabella per le definizione di concetti, etc...
\vspace{1em}
\rowcolors{1}{black!5}{black!5}
\setlength{\tabcolsep}{14pt}
\renewcommand{\arraystretch}{2}
\noindent
\begin{tabularx}{\textwidth}{@{}|P|@{}}
    \hline
    {\textbf{CONVERGENZA DI UNA SERIE}}\\
    \parbox{\linewidth}{Se la successione delle ridotte di una serie è convergente, si dice che la serie è convergente e il limite della successione delle ridotte prende il nome di \textbf{somma della serie}.\\
    In altre parole, se \textbf{esiste finito} il
    \[\lim_{n \to +\infty} s_n = s \in \mathbb{C}\]
    allora la serie si dice \textbf{convergente} e il limite $s$ si dice \textbf{somma della serie} e si scrive
    \[\sum_{n=0}^{+\infty} a_n = s\]
    \textbf{Attenzione}: Molto spesso si utilizza la notazione sopra esposta per indicare sia la serie stessa, sia la sua somma, per cui può essere fuorviante. Lo si può capire dal contesto: una serie potrebbe non essere convergente, e quindi non avere una somma.\vspace{3mm}}\\
    \hline
\end{tabularx}

\vspace{2em}
\noindent
\textbf{Esempio}: Se si considera $a_n=1, \forall n$, per cui
\[1+1+1+... = \sum_{n=0}^{n} 1\]
allora la somma parziale è $s_n=n+1$, ovvero una successione divergente a $+\infty$:
\[\lim_{n \to +\infty} s_n = +\infty\]
Ciò significa che la serie non converge, ma è \textbf{divergente}, per cui non ha nemmeno una somma.

\vspace{1em}
\noindent
\textbf{Osservazione}: Si osservi che la divergenza a $+\infty$ di una serie ha significato solamente quando i termini generali sono sul campo reale: se una serie ha termine generico nel campo complesso, non può essere divergente a $+\infty$, in quanto non esiste un limite infinito nel campo complesso (a meno che non si consideri il modulo).

\vspace{1em}
\noindent
\subsubsection{Divergenza di una serie}
Di seguito si espone la definizione di \textbf{divergenza di una serie}:

% Tabella per le definizione di concetti, etc...
\vspace{1em}
\rowcolors{1}{black!5}{black!5}
\setlength{\tabcolsep}{14pt}
\renewcommand{\arraystretch}{2}
\noindent
\begin{tabularx}{\textwidth}{@{}|P|@{}}
    \hline
    {\textbf{DIVERGENZA DI UNA SERIE}}\\
    \parbox{\linewidth}{Se la successione delle ridotte di una serie (a termine generale reale) è divergente, si dice che la serie è divergente; in questo caso, la serie non presenta una somma.\\
    In altre parole, se data $a_n \in \mathbb{R}, \forall n$, e posto
    \[\lim_{n \to +\infty} s_n = +\infty \text{ o } - \infty\]
    la serie si dice \textbf{divergente}.\vspace{3mm}}\\
    \hline
\end{tabularx}

\vspace{2em}
\noindent
\textbf{Esempio}: Se $a_n = a \in \mathbb{R}$ \textbf{costante}, allora la serie con termine generale $a_n$
\[a_0+a_1+a_2+...\]
è necessariamente 
\begin{itemize}
    \item divergente a $+\infty$ se $a > 0$
    \item divergente a $-\infty$ se $a < 0$
    \item convergente, con somma $0$, se $a = 0$
\end{itemize}
\textbf{Attenzione}: se $a \neq 0$, ma $a \in \mathbb{C} - \mathbb{R}$, si dice semplicemente che la serie \textbf{non converge} (non ha senso parlare di divergenza).


\vspace{1em}
\noindent
\subsubsection{Indeterminatezza di una serie}
Di seguito si espone la definizione di \textbf{serie indeterminata}:

% Tabella per le definizione di concetti, etc...
\vspace{1em}
\rowcolors{1}{black!5}{black!5}
\setlength{\tabcolsep}{14pt}
\renewcommand{\arraystretch}{2}
\noindent
\begin{tabularx}{\textwidth}{@{}|P|@{}}
    \hline
    {\textbf{SERIE INDETERMINATA}}\\
    \parbox{\linewidth}{Una serie si dice \textbf{indeterminata} se non converge e non diverge.\vspace{3mm}}\\
    \hline
\end{tabularx}

\vspace{2em}
\noindent
\textbf{Esempio 1}: Per quello che si è visto, una serie a termine generale costante, complesso e non reale, è indeterminata.

\vspace{1em}
\noindent
\textbf{Esempio 2}: Un esempio di serie a termini reali, ma indeterminata, è la \textbf{serie di Grandi}, definita così:
\[\sum_{n=0}^{+\infty} (-1)^n\]
per cui $s_0=(-1)^0=1$ e $s_1=a_0+a_1=1+(-1)^1=0$. Pertanto si ha che
\begin{itemize}
    \item $s_n=1$ se $n$ è pari
    \item $s_n=0$ se $n$ è dispari
\end{itemize}
Per cui si ha che
\[\lim_{n \to +\infty} s_0 = ? \text{ non esiste}\]
E per dimostrare che non esiste, si può semplicemente dimostrare che due sotto-successioni della successione delle somme parziali convergono a limiti diversi (ossia la sotto-successioni degli indici pari e quella dei dispari); infatti:
\begin{itemize}
    \item $\displaystyle{\lim_{k \to +\infty} s_{2k} = 1}$
    \item $\displaystyle{\lim_{k \to +\infty} s_{2k+1} = 0}$
\end{itemize}
per cui sono state ottenute due sotto-successioni che presentano limite differente: per il teorema dell'unicità del limite e il teorema del limite delle sotto-successioni di una successione, si conclude che la successione delle somme parziali è indeterminata.

\vspace{1em}
\noindent
\textbf{Osservazione}: La serie di Grandi è una serie che può essere usata per dimostrare l'esistenza di Dio, in quanto commutando fra di loro i differenti termini può essere fatta convergere a qualsiasi (o quasi) numero finito.\\
Se, infatti, si considerano le somme
\begin{itemize}
    \item $(1-1)+(1-1)+(1-1)+...=0$
    \item $1+(-1+1)+(-1+1)+...=1$
    \item $(1+1)+(-1+1)+(-1+1)=2$
\end{itemize}
si ottengono serie che convergono a qualunque valore (tranne uno). In generale, infatti, se una serie è indeterminata, si possono commutare gli addendi della stessa e ottenere la convergenza a qualunque numero.

\vspace{1em}
\noindent
\subsection{Serie geometrica}
Si è osservato che
\[\sum_{n=1}^{+\infty} \left(\frac{1}{2}\right)^n=1\]
per cui è ovvio che partendo con $n=0$, si ottiene
\[\sum_{n=0}^{+\infty} \left(\frac{1}{2}\right)^n=2\]
Più in generale, si fornisce di seguito la definizione di \textbf{serie geometrica}: 

% Tabella per le definizione di concetti, etc...
\vspace{1em}
\rowcolors{1}{black!5}{black!5}
\setlength{\tabcolsep}{14pt}
\renewcommand{\arraystretch}{2}
\noindent
\begin{tabularx}{\textwidth}{@{}|P|@{}}
    \hline
    {\textbf{SERIE GEOMETRICA}}\\
    \parbox{\linewidth}{Si dice \textbf{serie geometrica} di ragione $z \in \mathbb{C}$ la serie del tipo
    \[1+z+z^2+z^3+... \rightarrow \sum_{n=0}^{+\infty} z^n\]
    che, tuttavia, palesa un problema di fondo: se si sceglie $z=0$, naturalmente si incorre nell'ambiguità
    \[0^0 + 0^1 + ...\]
    ma $0^0$ è una scrittura che non ha significato. Tuttavia, in questo particolare caso, si considera $0^0=1$, in modo tale da essere coerenti con la scrittura $1+z+z^2+z^3+...$ impiegata in precedenza.\vspace{3mm}}\\
    \hline
\end{tabularx}

\vspace{1em}
\noindent
\textbf{Osservazione}: Data la serie seguente
\[\sum_{n=0}^{+\infty} z^n\]
per cui la ridotta è
\[s_n=1+z+z^2+...+z^n\]
che può anche essere riscritto come
\[s_n=1+z+z^2+...+z^n=1+z \cdot \left(1+z+...+z^{n-1}\right)\]
dove $1+z+...+z^{n-1}=s_{n-1}$. Da cui si evince che, sommando e sottraendo per la medesima quantità $z^n$, si ottiene
\[s_n = 1+z \cdot \left(\underbrace{1+z+...+z^{n-1}+z^n}_{s_n} - z^n\right)\]
che diviene, quindi:
\[s_n = 1 + z \cdot s_n - z^{n+1} \hspace{1em} \rightarrow \hspace{1em} s_n - z \cdot s_n = 1 - z^{n+1} \hspace{1em} \rightarrow \hspace{1em} s_n \cdot (1-z) = 1 - z^{n+1} \hspace{1em} \rightarrow \hspace{1em} s_n = \frac{1-z^{n+1}}{1-z}\]
posto $z \neq 1$ (ma il caso $z=1$ è facilmente risolubile, per quanto osservato nel caso di una serie a termine generale costante).\\
Di seguito si espone, quindi, il comportamento della serie geometrica a seconda della sua ragione $z$:

% Tabella per le definizione di concetti, etc...
\vspace{1em}
\rowcolors{1}{black!5}{black!5}
\setlength{\tabcolsep}{14pt}
\renewcommand{\arraystretch}{2}
\noindent
\begin{tabularx}{\textwidth}{@{}|P|@{}}
    \hline
    {\textbf{COMPORTAMENTO DELLA SERIE GEOMETRICA}}\\
    \parbox{\linewidth}{Per quanto osservato in precedenza, si ha che:
    \[\sum_{n=0}^{+\infty} z^n = \lim_{n \to +\infty} s_n = \lim_{n \to + \infty} \frac{1-z^{n+1}}{1-z}\]
    posto $z \neq 1$, che diviene
    \begin{itemize}
        \item $\displaystyle{\frac{1}{1-z}}$ se $\vert z \vert < 1$.
        \item non converge se $\vert z \vert > 1$, tuttavia, si può dire che
        \begin{itemize}
            \item se $z \in \mathbb{R}$ e $z \geq 1$, diverge a $+\infty$
            \item se $z \in \mathbb{C}$ e $\vert z \vert \geq 1$ (ovvero può essere anche un numero negativo), posto $z \notin \left]1,+\infty \right[$ (ossia diverso dal caso precedente), nel caso di $n$ pari si sommano quantità positive, nel caso di $n$ dispari si sommano quantità negative, per cui la serie oscilla e quindi è indeterminata.
        \end{itemize}
    \end{itemize}
    \vspace{1mm}}\\
    \hline
\end{tabularx}

\vspace{2em}
\noindent
\textbf{Osservazione}: Si osservi che la serie geometrica è l'unica per cui si riesce a calcolare la somma, in quanto è l'unica di cui è possibile esprimere la ridotta in modo generale.\\
Altrimenti, gestire le ridotte diviene molto complesso.

\vspace{1em}
\noindent
\textbf{Esempio}: Si consideri la seguente serie
\[\sum_{n=2}^{+\infty} \cos^{n}(1)\]
che è una serie geometrica di ragione $\cos(1)$, ove $\left \vert \cos(1) \right \vert < 1$, per cui converge. La somma di tale serie, quindi, è facilmente determinabile secondo quanto visto in precedenza, tenendo conto che $n$ parte da 2, per cui bisogna sottrarre $\cos^0(1)=1$ e $\cos^1(1)=\cos(1)$. Da ciò si evince che la serie converge a 
\[\frac{1}{1 - \cos(1)} - 1 - \cos(1) = \frac{1 - 1 + \cos(1) - \cos(1) + \cos^2(1)}{1 - \cos(1)} = \frac{\cos^2(1)}{1 - \cos(1)}\]

\vspace{1em}
\noindent
\textbf{Osservazione}: La somma della serie geometrica di ragione $z \in \mathbb{C}$ è indeterminata se $\vert z \vert > 1$, per quanto già visto.\\
Inoltre si ha che la serie
\[\sum_{n=1}^{+\infty} \left(\frac{2i + x}{4}\right)^n\]
è convergente se
\[\left \vert \frac{2i + x}{4}\right \vert < 1\]
ma ricordando come si calcola il modulo di un numero complesso si ottiene
\[\left \vert 2i + x \right \vert = \sqrt{4+x^2}\]
e quindi 
\[\sqrt{4+x^2} < 4 \hspace{1em} \rightarrow \hspace{1em} 4 + x^2 < 16 \hspace{1em} \rightarrow \hspace{1em} x^2 < 12 \hspace{1em} \rightarrow \hspace{1em} \vert x \vert < \sqrt{12} \hspace{1em} \rightarrow \hspace{1em} \vert x \vert < 2 \sqrt{3}\]
E poi, ovviamente, la serie di Grandi è il tipico esempio di serie indeterminata, per cui la sua somma non può essere definita.

\newpage
\noindent
\begin{center}
    5 Ottobre 2022
\end{center}
Una serie è costituita da $2$ successioni: la successione dei termini generali e la successione delle ridotte o somme parziali: quando si opera con le serie, risulta fondamentale distinguere le due successioni.\\
Una tra le serie più note è la serie geometrica, di ragione $z \in \mathbb{C}$, la quale converge se il modulo della ragione è minore di $1$. Non converge in caso contrario, ma in particolare
\begin{itemize}
    \item se la ragione $z$ è un numero reale, $z \in \mathbb{R}$, e $z \geq 1$, allora la serie diverge a $+\infty$;
    \item se la ragione $z$ è un numero complesso, con $\vert z \vert \geq 1$ e $z \notin ]1,+\infty[$, allora la serie è indeterminata.
\end{itemize} 
In generale, non si può parlare di divergenza a $+\infty$ o $-\infty$ in campo complesso, in quanto in esso è \textbf{assente la relazione d'ordine} e quindi non esiste un limite infinito.

\vspace{1em}
\noindent
\textbf{Esempio}: Si consideri l'esempio seguente:
\[\sum_{n=0}^{+\infty} \frac{\cos(n)}{2^n}\]
Tale serie presenta come termine generale
\[a_n = \frac{\cos(n)}{2^n}\]
ma è vero che $-1 \leq \cos(n) \leq 1$, per cui
\[-\frac{1}{2^n} \leq a_n \leq \frac{1}{2^n}\]
Per dimostrare che anche la serie in esame converge, è sufficiente considerare $s_n^-$ e $s_n^+$, rispettivamente la ridotta $n$-esima della serie geometrica di ragione $-\frac{1}{2}$ e $\frac{1}{2}$, come segue
\[s_n^- = -1 - \frac{1}{2} - ... - \frac{1}{2^n} \hspace{1em} \text{e} \hspace{1em} s_n^+ = 1 + \frac{1}{2} + ... + \frac{1}{2^n}\]
per cui
\[s_n^- \leq s_n \leq s_n^+\]
e per il \textbf{teorema del confronto esiste finito} il seguente limite
\[\lim_{n \to +\infty} s_n \in \mathbb{R}\]
e quindi la serie
\[\sum_{n=0}^{+\infty} \frac{\cos(n)}{2^n}\]
converge.

\vspace{1em}
\noindent
\subsection{Teorema del confronto per le serie}
Di seguito si espone il fondamentale \textbf{teorema del confronto per le serie}:

\vspace{1em}
\noindent
\begin{theorem} \textbf{Teorema del confronto per le serie}\\
    Siano $a_n,b_n,c_n \in \mathbb{R}$ tali che $a_n \leq b_n \leq c_n, \forall n$ (anche se sarebbe sufficiente richiedere che ciò sia vero \textbf{definitivamente}, ossia $\exists n_0 \in \mathbb{N}$ tale che la disuguaglianza di cui sopra è valida $\forall n \geq n_0$). Siano convergenti le serie
    \[\sum a_n \hspace{1em} \text{e} \hspace{1em} \sum c_n\]
    allora è convergente anche la serie
    \[\sum b_n\]
    ed è tale la stima della somma della serie:
    \[\sum a_n \leq \sum b_n \leq c_n\]
    che è una stima valida $\forall n$, oppure $\forall n \geq n_0$ (a seconda che sia stato richiesto $\forall n$, oppure definitivamente).
\end{theorem}

\vspace{1em}
\noindent
\textbf{Osservazione}: Si osservi il caso particolare per cui $a_n=0, \forall n$ (ossia il caso in cui la serie con termine generale $b_n$ è a termini positivi, cioè una serie per cui tutti i termini della successione dei termini generali sono positivi), allora è sufficiente che la serie con termine generale $c_n$ converga per concludere la convergenza.\\
Similmente, se $c_n=0, \forall n$ (ossia la serie con termine generale $b_n$ è a termini negativi, vale a dire una serie per cui tutti i termini della successione dei termini generali sono negativi), è sufficiente che la serie con termine generale $a_n$ converga per concludere la convergenza.\\
In questi casi, infatti, è sufficiente considerare un limitazione superiore (o inferiore, rispettivamente) per concludere la convergenza.

\vspace{1em}
\noindent
\textbf{Osservazione}: È facile capire che \textbf{il carattere di una serie non dipende da quello che accade su un numero finito di termini}, in quanto
\[\sum_{n=k}^{+\infty} a_n \hspace{1em} \text{ e } \hspace{1em} \sum_{n=0}^{+\infty} a_n\]
differiscono solamente per $k$ termini, ove $k$ è una \textbf{costante}.

\vspace{1em}
\noindent
\textbf{Esempio}: Si consideri la serie
\[\sum_{n=0}^{+\infty} \frac{1}{2^n} e^{100-n^2}\]
Si può facilmente capire che
\[e^{100-n^2} \leq 1 \hspace{1em} \text{ se } \hspace{1em} n \geq 10\]
per cui
\[\frac{1}{2^n} e^{100-n^2} \leq \frac{1}{2^n}  \hspace{1em} \text{ se } \hspace{1em} n \geq 10\]
Pertanto, essendo essa a termini positivi e maggiorata definitivamente, la serie di partenza converge per il teorema del confronto. Tuttavia, la stima seguente
\[\sum_{n=0}^{+\infty} \frac{1}{2^n} e^{100-n^2} \leq 2\]
ove $2$ è la somma della serie geometrica, è vera solamente definitivamente, per $n \geq 10$. Per avere una stima della somma più accurata, naturalmente, è possibile considerare quello che accade per i primi $9$ termini, per cui:
\[\sum_{n=0}^{+\infty} \frac{1}{2^n} e^{100-n^2} < a_0+a_1+\dots+a_9+2\]
dove $a_0+a_1+\dots+a_9$ sono i primi $9$ termini della serie stessa. Ma per migliorare la stima è possibile anche considerare i primi $9$ termini della serie geometrica, da cui
\[\sum_{n=0}^{+\infty} \frac{1}{2^n} e^{100-n^2} < a_0+a_1+\dots+a_9+\left(2-1-\frac{1}{2}-\dots-\frac{1}{2^9}\right)\]

\vspace{1em}
\noindent
\textbf{Esempio}: Si consideri la serie seguente:
\[\sum_{n=1}^{+\infty} \cos \left(\frac{1}{n}\right)\]
Essa naturalmente diverge, in quanto il limite per $n \to +\infty$ del suo termine generale è
\[\lim_{n \to +\infty} \cos \left(\frac{1}{n}\right)=1\]
ossia, per $n$ molto grande, nella serie si somma sempre $1$, per cui diverge. Infatti, affinché una serie converga, il suo termine generale deve essere infinitesimo.

\vspace{1em}
\noindent
\begin{theorem} \textbf{Condizione necessaria per la convergenza}\\
    Sia
    \[\sum_{n=0}^{+\infty} a_n\]
    una serie \textbf{convergente} (in generale a termini complessi), allora
    \[\lim_{n \to \infty} a_n=0\]
    ossia la successione dei termini generali è \textbf{infinitesima}.
\end{theorem}

% Formattazione per la dimostrazione, etc.
\vspace{2em}
\noindent
\normalfont \normalsize
\textsc{Dimostrazione}: Si consideri la ridotta di indice $n+1$, ossia
\[s_{n+1} = s_n + a_{n+1} \hspace{1em} \text{ tale per cui } \hspace{1em} a_{n+1} = s_{n+1} - s_n\]
Siccome la serie è convergente per ipotesi ($s_{n+1}$ e $s_n$ convergono allo stesso limite), per la linearità del limite, il limite della differenza è uguale alla differenza dei limiti, per cui:
\[\lim_{n \to +\infty} a_{n+1} = s_{n+1} - s_n = 0\]

\vspace{1em}
\noindent
\textbf{Osservazione}: Si osservi che la condizione per la convergenza esposta in precedenza è necessaria, ma non sufficiente. Infatti, esistono delle serie
\[\sum a_n\]
non convergenti, dove il
\[\lim_{n \to +\infty} a_n = 0\]
per questo si parla di condizione necessaria, e non sufficiente. Infatti è importante definire con quale velocità la successione dei termini generali vada a $0$: se è troppo lenta, nonostante sia infinitesima, la serie associata divergerà.

\vspace{1em}
\noindent
\subsection{Serie armonica}
Si consideri la serie seguente
\[\sum_{n=1}^{+\infty} \frac{1}{n}\]
che prende il nome di \textbf{serie armonica}. Per studiarne il comportamento, è sufficiente capire che \textbf{ogni serie può essere considerata come un integrale generalizzato}. Infatti, per definizione di integrale generalizzato di una funzione definita su una semiretta reale localmente integrabile:
\[\int_a^{+\infty} f(x) \dif x = \lim_{b \to +\infty} \int_a^b f(x) \dif x\]
allora se si considera la serie $a_1+a_2+a_3+\dots+a_n$, si definisce una funzione $f$ dipendente dalla serie stessa:
\[f : [1,+\infty[ \hspace{0.5em} \longmapsto \mathbb{R}\]
nel modo seguente: essendo una successione una funzione (definita sui numeri naturali), la funzione $f$ deve interpolare i valori della successione dei termini generali, assumendo il valore costante $a_n$ quando $x \in [n,n+1[$, come nel seguito:
\[f(x)=a_n \hspace{1em} \text{ se } \hspace{1em} x \in [n,n+1[, \hspace{1em} \forall n \geq 1\]
ottenendo una funzione che rappresenta la successione degli $a_n$ sotto forma di funzione.\\
Se $f$ è la successione degli $a_n$, la serie con termine generale $a_n$ non è altro che l'integrale generalizzato di tale funzione. Infatti, si ha che
\[\int_{n}^{n+1} f(x) \dif x = a_n \cdot (n+1-n) = a_n\]
per cui è ovvio che
\[s_n=a_1+a_2+\dots+a_n=\int_1^{n+1} f(x) \dif x\]
Se la funzione $f$ è integrabile (ossia esiste il limite dell'integrale di cui sopra), allora
\[\int_1^{+\infty} f(x) \dif x = \lim_{b \to +\infty} \int_1^b f(x) \dif x\]
e per quanto appena osservato,
\[s_n=a_1+a_2+\dots+a_n=\int_1^{n+1} f(x) \dif x \hspace{1em} \text{ allora } \hspace{1em} \lim_{n \to +\infty} s_n = \lim_{n \to +\infty} \int_1^{n+1} f(x) \dif x\]
per cui, per il teorema del limite delle successioni, ogni successione in cui $n$ tende a $+\infty$, avrà lo stesso limite della funzione $f$, ossia
\[\lim_{n \to +\infty} s_n = \lim_{n \to +\infty} \int_1^{n+1} f(x) \dif x = \int_1^{+\infty} f(x) \dif x\]
Pertanto, se la funzione $f$ così definita è integrabile e l'integrale ha un valore finito, allora la serie è convergente e la somma della serie è il valore di tale integrale.

\vspace{1em}
\noindent
\textbf{Osservazione}: Si osservi che se la serie converge, per cui
\[\lim_{n \to +\infty} \int_1^{n+1} f(x) \dif x = s\]
è anche vero che $f$ è integrabile, ovvero
\[\int_1^{+\infty} f(x) \dif x = s\]
Ciò è vero in quanto la serie converge, e per la condizione necessaria vista in precedenza,
\[\lim_{n \to +\infty} a_n = 0\]
Pertanto, studiando l'integrale
\[\int_1^b f(x) \dif x\]
presa la parte intera di $b$, ossia $[b]=n$, essendo $b < n+1$ (in quanto la sua parte intera è $n$), si evince che 
\[\int_1^b f(x) \dif x = \int_1^n f(x) \dif x + \int_n^b f(x) \dif x\]
Dovendo studiare il limite per $b \to + \infty$ di tale integrale, è molto utile scomporlo in questo modo. Così facendo, siccome la serie converge, si ha che
\[\lim_{n \to +\infty} \int_1^n f(x) \dif x = s\]
mentre
\[\left \vert \int_n^b f(x) \dif x \right \vert = \left \vert a_n \cdot (b-n) \right \vert \leq \left \vert a_n \right \vert\]
in quanto $b<n+1$, per cui $b-n<1$, essendo $[b]=n$. Ma siccome la serie converge, allora il limite del termine generale è $0$, quindi
\[\lim_{n \to +\infty} \int_n^b f(x) \dif x \leq \lim_{n \to +\infty} a_n = 0\]
per cui, per la linearità del limite, si ha
\[\lim_{n \to +\infty} \int_1^b f(x) \dif x = \lim_{n \to +\infty} \int_1^n f(x) \dif x + \lim_{n \to +\infty} \int_n^b f(x) \dif x = s+0=s\]
come esposto da teorema seguente:

\vspace{2em}
\noindent
\begin{theorem}
    Sia $a_1+a_2+\dots+a_n$ una serie e sia $f$ la funzione associata definita come
    \[f(x)=a_n \hspace{1em} \text{ se } \hspace{1em} x \in [n,n+1[, \hspace{1em} \forall n \geq 1\]
    allora $f$ è integrabile in senso generalizzato sull'intervallo $[1,+\infty[$ \textbf{se e solo} se la serie converge. In questo caso si ha che la somma della serie è uguale al valore dell'integrale generalizzato, per cui
    \[\sum_{n=1}^{+\infty} a_n = \int_1^{+\infty} f(x) \dif x\]
\end{theorem}

\vspace{1em}
\noindent
\textbf{Osservazione}: Tale risultato è fondamentale per studiare il carattere della serie armonica. Infatti, se si considera la funzione
\[g(x)=\frac{1}{x}\]
essa non è integrabile in senso generalizzato sull'intervallo $[1,+\infty[$. Allora, presa $f(x)$ la funzione definita a tratti rispetto alla serie armonica, è facle capire che
\[g(x) \leq f(x), \hspace{1em} \forall x \in [1,+\infty[\]
Dal momento che $g(x)$ non è integrabile, non lo è nemmeno la $f$ (per il teorema del confronto degli integrali generalizzati).\\
Ma siccome, per il teorema precedentemente esposto, è noto che una serie converge se e solo se la funzione $f$ ad essa associata converge, si capisce immediatamente che la serie
\[\sum_{n=1}^{+\infty} \frac{1}{n}\]
non converge. Essendo una serie a termini positivi, per l'aut-aut si vedrà immediatamente che, non convergendo, dovrà necessariamente essere divergente a $+\infty$.

\vspace{1em}
\noindent
\subsubsection{Serie armonica generalizzata}
È noto che la serie armonica non converge. Non sorprende, però, sapere che tale serie è divergente a $+\infty$, ovvero
\[\sum_{n=1}^{+\infty} \frac{1}{n} = + \infty\]
come conseguenza diretta dell'aut-aut. Pertanto, se si considera
\[\sum_{n=1}^{+\infty} \frac{1}{\sqrt{n}}\]
è evidente capire che
\[\frac{1}{\sqrt{n}} \geq \frac{1}{n}, \hspace{1em} \forall n \geq 1\]
per cui, per il teorema del confronto, diverge a $+\infty$. Ciò risulta vero per ogni
\[\frac{1}{n^\alpha} \geq \frac{1}{n},\hspace{1em} \forall n \geq 1 \hspace{1em} \text{ se } 0 < \alpha \leq 1\]
Nel caso $\alpha > 1$, invece, è possibile studiare l'integrale generalizzato associato, da cui:
\[\int_1^{+\infty} \frac{1}{x^\alpha} \dif x = \left[\frac{1}{-\alpha+1} \cdot x^{-\alpha+1}\right]_1^{+\infty} = \frac{1}{\alpha-1}\]
Tuttavia, ciò non risulta essere sufficiente per dimostrare che la serie converge. Infatti, in questo caso, si è studiato l'integrale generalizzato di una funzione $g(x)$, ben diversa dalla funzione $f$ definita a tratti in precedenza.\\
Se ora si impiegasse la funzione $f$ definita in precedenza (da $n$ a $n+1$), siccome essa sarà inevitabilmente maggiore della funzione $g$ (di cui è nota l'integrabilità), ovvero $f(x) \geq g(x)$, non è possibile stabilire se essa sia integrabile o meno tramite il criterio del confronto per l'integrale generalizzato.\\
Per tale ragione si definisce
\[h(x)=a_n \hspace{1em} \text{ se } \hspace{1em} x \in ]n-1,n]\]
tale per cui $h(x) \leq g(x), \hspace{0.5em} \forall n \geq 1$. Allora è noto che
\[\int_{n-1}^n h(x) \dif x = a_n\]
Da ciò segue che
\[\int_1^{+\infty} h(x) \dif x=a_2+a_3+\dots=\sum_{n=2}^{+\infty} a_n\]
che parte da $n=2$, per come è stata definita $h(x)$. Pertanto, si ha che
\[\int_1^{+\infty} h(x) \dif x \leq \int_1^{+\infty} \frac{1}{x^\alpha} \dif x = \frac{1}{\alpha-1}\]
e quindi, per il teorema del confronto dell'integrale generalizzato, la funzione $h$ è integrabile. Inoltre, per il teorema precedentemente esposto, siccome la funzione $h$ associata alla serie è integrabile, la serie armonica generalizza converge; non solo, la somma della serie è
\[\sum_{n=1}^{+\infty} \frac{1}{n^\alpha} \leq \frac{1}{\alpha-1}+1\]

% Tabella per le definizione di concetti, etc...
\vspace{1em}
\rowcolors{1}{black!5}{black!5}
\setlength{\tabcolsep}{14pt}
\renewcommand{\arraystretch}{2}
\noindent
\begin{tabularx}{\textwidth}{@{}|P|@{}}
    \hline
    {\textbf{COMPORTAMENTO DELLA SERIE ARMONICA GENERALIZZATA}}\\
    \parbox{\linewidth}{La serie armonica generalizzata
    \[\sum_{n=1}^{+\infty} \frac{1}{n^\alpha}\]
    con $\alpha>$ è
    \begin{itemize}
        \item divergente a $+\infty$ se $\alpha \in ]0,1]$
        \item convergente se $\alpha>1$ con somma
        \[s \leq 1 + \frac{1}{\alpha - 1} = \frac{\alpha}{\alpha-1}\]
        dal momento che l'integrale
        \[\int_1^{+\infty} h(x) \dif x = \sum_{n=2}^{+\infty} \frac{1}{n^\alpha} \hspace{1em} \text{ in particolare } \hspace{1em} \int_1^{+\infty} \frac{1}{x^\alpha} = \frac{1}{\alpha-1} \geq \sum_{n=2}^{+\infty} \frac{1}{n^\alpha}\]
        e siccome parte da $n=2$, è necessario aggiungere $1$, da cui la disuguaglianza esposta.
    \end{itemize}
    \vspace{1mm}}\\
    \hline
\end{tabularx}

\vspace{2em}
\noindent
\textbf{Esercizio 1}: Si consideri la serie
\[\sum_{n=2}^{+\infty} \frac{1}{\log(n)}\]
che, ovviamente, diverge in quanto
\[\frac{1}{\log(n)} \geq \frac{1}{n}, \hspace{1em} \forall n \geq e\]
e siccome $\dfrac{1}{n}$ diverge, per il teorema del confronto, diverge anche $\dfrac{1}{\log(n)}$.

\vspace{2em}
\noindent
\textbf{Esercizio 2}: Si consideri la serie
\[\sum_{n=1}^{+\infty} \frac{1}{n \cdot (\log(n))^\alpha}\]
Per capire se essa diverga o meno, si considera l'integrale
\[\int_1^{+\infty} \frac{1}{n \cdot (\log(n))^\alpha} \dif x = \lim_{b \to +\infty} \int_1^b \frac{1}{n \cdot (\log(n))^\alpha} \dif x = \lim_{b \to +\infty} \left[\frac{1}{-\alpha+1} \log^{-\alpha+1}(x)\right]_1^b\]
in cui
\begin{itemize}
    \item se $\alpha > 1$, allora la funzione non è integrabile in senso generalizzato;
    \item se $\alpha = 1$, l'integrale è nullo e la funzione è integrabile in senso generalizzato.
\end{itemize}

\vspace{2em}
\noindent
\textbf{Esercizio 3}: Si consideri la serie
\[\sum_{n=2}^{+\infty} \frac{\arctan(n^2)}{n \cdot \sqrt{n}}\]
È ovvio che il numeratore è limitato, in quanto
\[\arctan(n^2) \leq \frac{\pi}{2}, \hspace{1em} \forall n\]
e quindi si evince che
\[\left \vert \frac{\arctan(n^2)}{n \cdot \sqrt{n}} \right \vert \leq \frac{\pi}{2} \frac{1}{n \cdot \sqrt{n}}\]
ove
\[\sum_{n=2}^{+\infty}  \frac{1}{n \cdot \sqrt{n}}\]
è una serie armonica generalizzata di ragione $\dfrac{3}{2} > 1$ che converge. Per il criterio del confronto per le serie, anche la serie di partenza converge.

\vspace{1em}
\noindent
\subsection{Serie a termini (reali) positivi}
Si consideri una serie a termini (reali) positivi, tale che $a_n \geq 0, \forall n$ (anche se sarebbe sufficiente \textbf{definitivamente}, ossia da un certo $n$ in poi).\\
Allora, per il \textbf{teorema dell'Aut-Aut}, tale serie o converge, o diverge, ma non può essere indeterminata.\\
Ciò spiega perché la serie armonica diverga a $+\infty$, in quanto si è dimostrato che non converge ed è una serie a termini (reali) positivi; naturalmente, il teorema dell'Aut-Aut si aggiunge al teorema del confronto.\\
Un altro importante criterio è l'ordine di infinitesimo che, tuttavia, non risulta efficace quando si considerano serie il cui termine generale presenta un ordine infrareale, ossia maggiore di $\alpha$, ma più piccolo di $\alpha+\epsilon, \hspace{0.5em} \forall \epsilon > 0$.

\newpage
\noindent
\begin{center}
    7 Ottobre 2022
\end{center}
\noindent
Dopo aver analizzato la condizione necessaria per la convergenza, è stato anche considerato il fatto che una serie può essere sempre considerata come un integrale generalizzato. Un esempio fondamentale di serie di confronto è anche la serie armonica.\\
Di seguito si espongono alcuni teoremi fondamentali per la convergenza/divergenza di una serie.

\vspace{1em}
\noindent
\subsection{Teorema dell'Aut-Aut per le serie a termini (reali) positivi}
Si supponga che la serie
\[a_1+a_2+\dots+a_n+\dots=\sum_{n=1}^{+\infty} a_n\]
abbia termini positivi ($a_n > 0)$ o al più non negativi ($a_n \geq 0$). Allora essa converge o diverge. In altre parole, una serie a termini non negativi non può essere indeterminata.

\vspace{2em}
\noindent
\normalfont \normalsize
\textsc{Dimostrazione}: Supposto $a_n \geq 0, \forall n$ (anche se sarebbe sufficiente richiedere definitivamente), la successione delle ridotte è \textbf{crescente (anche in senso debole)}, tale per cui
\[s_{n+1}=s_n+a_{n+1} \geq s_n\]
Per il \textbf{teorema di esistenza del limite delle successioni monotone}, la successione delle ridotte ammette limite, ed esso è
\[\lim_{n \to +\infty} s_n = \text{ sup } \{s_n : n \in \mathbb{N}^+\}\]
Pertanto
\begin{itemize}
    \item se la successione delle ridotte è superiormente limitata, ovvero $\text{sup } \{s_n\} \in \mathbb{R}$, la serie è ovviamente convergente.
    \item se la successione delle ridotte è superiormente illimitata, per cui $\text{sup } \{s_n\} = +\infty$, la serie diverge a $+\infty$.
\end{itemize}
In ogni caso, però, \textbf{la serie non può essere indeterminata}.

\vspace{1em}
\noindent
\textbf{Osservazione}: Naturalmente la stessa cosa vale anche per successioni a termini negativi. L'importante è che sia verificata la condizione $a_n \geq 0$ oppure $a_n \leq 0$ infinitesimo.

\vspace{1em}
\noindent
\subsection{Criterio dell'ordine di infinitesimo per le serie a termini positivi}
Il teorema dell'Aut-Aut permette di dimostrare anche un altro importante criterio:

\begin{theorem}\textbf{Criterio dell'ordine di infinitesimo per le serie a termini positivi}\\
    Sia
    \[\sum_{n=0}^{+\infty} a_n\]
    una serie a termini positivi con termine generale infinitesimo
    \[\lim_{n \to +\infty} a_n = 0\]
    allora
    \begin{itemize}
        \item se esiste $\alpha > 1 \vert \text{ ord } a_n \geq \alpha$, la serie converge
        \item se esiste $\alpha > 1 \vert \text{ ord } a_n \leq 1$, la serie diverge
    \end{itemize}
\end{theorem}

\vspace{2em}
\noindent
\normalfont \normalsize
\textsc{Dimostrazione}: Supposto che la successione $a_n$ abbia come ordine di infinitesimo $\alpha$, con $\alpha > 1$, ossia
\[\lim \left \vert \frac{a_n}{\frac{1}{n^\alpha}}\right \vert = l \hspace{1em} \text{ posto } \hspace{1em} l \neq 0\]
allora, per definizione stessa di limite,
\[\forall \epsilon > 0, \exists n_\epsilon \in \mathbb{N} \vert \forall n > n_\epsilon \text{ si ha} n^\alpha < l + \epsilon\]
Per comodità, si sceglie $\epsilon = 1$, da cui
\[n^\alpha a_n < l+1\]
Ciò consente di affermare che $\forall n > n_\epsilon$ si ha che
\[0 \leq a_n \leq (l+1) \cdot \frac{1}{n^\alpha}\]
In questo modo si sta confrontando il termine generale $a_n$ con il termine generale della serie armonica generalizzata. Per il teorema del confronto, siccome definitivamente
\[a_n \leq (l+1) \cdot \frac{1}{n^\alpha}\]
e la serie armonica converge, in quanto $\alpha > 1$ ... continua ...

\vspace{1em}
\noindent
Supposto, ora, $\text{ord } a_n \leq 1$, si dimostri che la serie
\[\sum_{n=1}^{+\infty}a_n\]
diverge.\\
Il fatto che $\text{ord } a_n \leq 1$, significa che
\[\lim_{n \to +\infty} \frac{a_n}{\frac{1}{n}} = l\]
per cui se $l \in \mathbb{R} - \{0\}$ significa che $\text{ord } a_n = 1$, se $l = +\infty$, $\text{ord } a_n < 1$.\\
Nell'ipotesi che $l \in \mathbb{R} - \{0\}$, ovvero
\[\lim_{n \to +\infty} n \cdot a_n = l \in \mathbb{R} - \{0\}\] allora, per la definizione stessa di limite
\[\forall \epsilon > 0, \exists n_\epsilon \in \mathbb{N} \vert \forall n > n_\epsilon : \left \vert a_n-l \right \vert < \epsilon\]
Scelto, per comodità, $\epsilon=\frac{l}{2}$, e quindi ... continua ...
\[\]

\vspace{1em}
\noindent
\textbf{Osservazione}: In particolare, se $\exists \alpha \in \mathbb{R}, \alpha>1$, e si ha
\begin{itemize}
    \item $\text{ord } a_n > \alpha$, la serie converge
    \item $\text{ord } a_n \leq 1$, la serie diverge
\end{itemize}
Tuttavia, sapere che $\text{ord } a_n > 1$ non fornisce informazioni

\vspace{1em}
\noindent
\textbf{Esercizio 1}: La serie
\[\sum \frac{5n + \cos(n)}{3 + 2n^3}\]
è ovviamente convergente, in quanto $\text{ord } a_n = 2 > 1$.

\vspace{1em}
\noindent
\textbf{Esercizio 2}: La serie
\[\sum \frac{2\sqrt{n}}{n^2+n+1}\]
è ovviamente convergente, in quanto $\text{ord } a_n = \frac{3}{2} > 1$.

\vspace{1em}
\noindent
\textbf{Esercizio 3}: La serie
\[\sum \log \left(1-\frac{1}{n}\right)\]
non converge. La serie è a termini negativi, tuttavia si può fare
\[-\lim_{n \to +\infty} -\frac{\log \left(1-\frac{1}{n}\right)}{\frac{1}{n}}=1\]
per cui $\text{ord } a_n = 1$.

\vspace{1em}
\noindent
\textbf{Esercizio 4}: La serie
\[\sum 1 - \cos \left(\frac{1}{n}\right)\]
è ovviamente convergente, in quanto $\text{ord } a_n = 2 > 1$.

\vspace{1em}
\noindent
\textbf{Esercizio 5}: La serie
\[\sum \frac{2^n}{(\log(n))^n} = \sum \left(\frac{2}{\log(n)}\right)^n\]
è ovviamente convergente, in quanto
\[\frac{2}{\log(n)} < \frac{2}{3} \rightarrow \log(n) > 3\]
per $n > e^3$, ma l'importante è che accada definitivamente, per cui la serie converge per confronto con la serie geometrica.

\vspace{1em}
\noindent
\textbf{Esercizio 6}: La serie
\[\sum \]

\vspace{1em}
\noindent
\subsection{Criterio del rapporto}
Presa una serie a termini positivi, ma non nulli (in quanto bisogna dividere per il termine $a_n$), per cui $a_n>0$, come la seguente
\[\sum_{n=0}^{+\infty} a_n\]
tale per cui
\[\exists \lim_{n \to +\infty} \frac{a_{n+1}}{a_n} = k\]
Allora
\begin{itemize}
    \item se $k < 1$ la serie converge
    \item se $k = 1$ la serie diverge
    \item se $k > 1$ non è possibile dire nulla in merito al carattere della serie
\end{itemize}

\vspace{2em}
\noindent
\normalfont \normalsize
\textsc{Dimostrazione 1}: Si consideri 
\[\lim_{n \to +\infty} \frac{a_{n+1}}{a_n}=k\]
con $k<1$. Allora, per la definizione di limite
\[\exists \epsilon > 0, \exists n_\epsilon \in \mathbb{N} \vert \forall n > n_\epsilon, k - \epsilon < \frac{a_{n+1}}{a_n} < k+\epsilon\]
preso un $\epsilon$ sufficientemente piccolo
\[\frac{a_{n+1}}{a_n} < k+\epsilon < 1 \hspace{1em} \rightarrow \hspace{1em} a_{n+1} < (k+\epsilon) \cdot a_n\]
E avendo supposto $a_n>0$, è ovvio che
\[0 < a_n < \dots\]
senza perdita di generalità (in quanto si richiederebbe $\forall n > n_\epsilon$), è possibile supporre che
\[a_{n+1} < (k+\epsilon) \cdot a_n, \forall n\]
per cui, si ha che
\[a_n < (k+\epsilon)^n \cdot a_0\]
e, quindi, essendo $a_0$ costante, per il teorema del confronto, la serie
\[\sum_{n=1}^{+\infty}\]
converge.

\vspace{2em}
\noindent
\normalfont \normalsize
\textsc{Dimostrazione 2}: Si consideri 
\[\lim_{n \to +\infty} \frac{a_{n+1}}{a_n}=k\]
con $k>1$. ... continua ... Allora, per la definizione di limite
\[\exists \epsilon > 0, \exists n_\epsilon \in \mathbb{N} \vert \forall n > n_\epsilon, k - \epsilon < \frac{a_{n+1}}{a_n} < k+\epsilon\]
preso un $\epsilon$ sufficientemente piccolo
\[\frac{a_{n+1}}{a_n} < k+\epsilon < 1 \hspace{1em} \rightarrow \hspace{1em} a_{n+1} < (k+\epsilon) \cdot a_n\]
E avendo supposto $a_n>0$, è ovvio che
\[0 < a_n < \dots\]
senza perdita di generalità (in quanto si richiederebbe $\forall n > n_\epsilon$), è possibile supporre che
\[a_{n+1} < (k+\epsilon) \cdot a_n, \forall n\]
per cui, si ha che
\[a_n < (k+\epsilon)^n \cdot a_0\]
e, quindi, essendo $a_0$ costante, per il teorema del confronto, la serie
\[\sum_{n=1}^{+\infty}\]
converge.

\vspace{1em}
\noindent
\textbf{Esempio}: Si consideri la serie
\[\sum_{n=1}^{+\infty} \frac{n^n}{(n!)^2}\]
Allora, applicando il teorema del rapporto
\[\frac{a_{n+1}}{a_n} = \frac{\frac{(n+1)^{n+1}}{[(n+1)!]^2}}{\frac{n^n}{(n!)^2}}\]
Che può essere riscritto come
\[\lim_{n \to +\infty} (n+1) \cdot \left(\frac{n+1}{n}\right)^n \cdot \frac{(n!)^2}{(n+1)^2 \cdot (n!)^2} = 0\]
e siccome $0<1$, la serie converge. Non solo, siccome la serie converge, la successione delle somme parziali è infinitesima.

\vspace{1em}
\noindent
\subsection{Criterio della radice $n$-esima}
Sia una serie a termini positivi, con $a_n \geq 0, \forall n$ (anche se sarebbe sufficiente richiederlo definitivamente). Supposto che esiste
\[\lim_{n \to +\infty} \left(\sqrt[n]{a_n}\right)=l\]
Allora si considerano le seguenti casistiche
\begin{itemize}
    \item se $l>1$ la serie diverge
    \item se $l<1$ la serie converge
    \item se $l=1$ non si può dire nulla sul carattere della serie
\end{itemize}

\vspace{2em}
\noindent
\normalfont \normalsize
\textsc{Dimostrazione 1}: Si consideri il caso in cui 
\[\lim_{n \to +\infty} \left(\sqrt[n]{a_n}\right)=l\]
com $l>1$, per la definizione di limite
\[\forall \epsilon > 0, \exists n_\epsilon \in \mathbb{N} \vert \forall n > n_\epsilon, \left \vert \sqrt[n]{a_n} - l \right \vert < \epsilon\]
da cui $\sqrt[n]{a_n} > l - \epsilon$, per cui posto $\epsilon = 1$ si ha che, definitivamente $a_n>1$ e quindi la serie non può convergere.

\vspace{2em}
\noindent
\normalfont \normalsize
\textsc{Dimostrazione 2}: Si consideri il caso in cui 
\[\lim_{n \to +\infty} \left(\sqrt[n]{a_n}\right)=l\]
com $l<1$, per la definizione di limite
\[\forall \epsilon > 0, \exists n_\epsilon \in \mathbb{N} \vert \forall n > n_\epsilon, \left \vert \sqrt[n]{a_n} - l \right \vert < \epsilon\]
e, prendendo $0 < \epsilon < 1-l$, si evince che
\[\sqrt[n]{a_n} < l+\epsilon < 1 \hspace{1em} \rightarrow \hspace{1em} a_n < (l+\epsilon)^n\]
e siccome si è preso $\vert q \vert < 1$, per confronto con la serie converge.

\vspace{1em}
\noindent
\textbf{Esempio}: Si consideri la serie seguente
\[\sum_{n=1}^{+\infty} \frac{1}{n^2}\]
applicando il criterio del rapporto si ha
\[\lim_{n \to +\infty} \frac{n^2}{(n+1)^2} = 1\]
per cui per tale criterio non è possibile dire nulla, ma è noto che la serie converge.\\
Analogamente si ha che il carattere della serie
\[\sum_{n=1}^{+\infty} \frac{1}{n}\]
non può essere determinato con il criterio del rapporto, in quanto
\[\lim_{n \to +\infty} \frac{n}{n+1} = 1\]
ma è noto che tale serie diverge.

\vspace{1em}
\noindent
\subsection{Serie a termini qualsiasi}
Se si considera una serie a termine generale qualsiasi
\[\sum a_n, \hspace{1em} \text{ con } \hspace{1em} a_n \in \mathbb{C}\]
non è possibile dire molto sul suo carattere. Tuttavia, ad essa è possibile associare la serie
\[\sum \vert a_n \vert\]
che è una serie a termine generale positivo. Da ciò segue anche la definizione di \textbf{serie assolutamente convergente}:

% Tabella per le definizione di concetti, etc...
\vspace{1em}
\rowcolors{1}{black!5}{black!5}
\setlength{\tabcolsep}{14pt}
\renewcommand{\arraystretch}{2}
\noindent
\begin{tabularx}{\textwidth}{@{}|P|@{}}
    \hline
    {\textbf{SERIE ASSOLUTAMENTE CONVERGENTE}}\\
    \parbox{\linewidth}{Una serie
    \[\sum a_n\]
    si dice \textbf{assolutamente convergente}, se è convergente la serie
    \[\sum \vert a_n \vert\]
    \vspace{-1mm}}\\
    \hline
\end{tabularx}

\begin{theorem}
    Una serie assolutamente convergente è convergente.
\end{theorem}

\vspace{2em}
\noindent
\normalfont \normalsize
\textsc{Dimostrazione}: Si consideri il caso in cui $a_n \in \mathbb{R}$, allora, per definizione di parte positiva e parte negativa si ha
\[a_n^+ = \left\{\begin{array}{lll}
    a_n & \text{se} & a_n \geq 0\\
    0   & \text{se} & a_n < 0\\
\end{array}
\right.\]
\[a_n^- = \left\{\begin{array}{lll}
    -a_n & \text{se} & a_n < 0\\
    0    & \text{se} & a_n \geq 0\\
\end{array}
\right.\]
ma ciò significa che $a_n=a_n^+-a_n^-$, mentre $\vert a_n \vert a_n^+ + a_n^-$, ma è ance vero che
\[0 \leq a_n^+ \leq \vert a_n \vert\]
\[0 \leq a_n^- \leq \vert a_n \vert\]

\vspace{1em}
\noindent
Se si considera un numero complesso 
Ma ciò significa

\vspace{1em}
\noindent
\textbf{Osservazione}: Tuttavia, non è vero il viceversa, come nel caso della \textbf{serie di Leibniz}... continua ...

\vspace{1em}
\noindent
\subsection{Limiti di successioni in $\mathbb{C}$}
Sia $(z_n)_n$ una successione in $\mathbb{C}$, con $\gamma \in \mathbb{C}$, si dirà che
\[\lim_{n \to +\infty} z_n = \gamma\]
se
\[\forall \epsilon > 0, \exists n_\epsilon \in \mathbb{N} \vert \forall n \geq n_\epsilon, \vert z_n - \gamma \vert < \epsilon\]
in cui è da intendersi $\vert \dots \vert$ come modulo di un numero complesso. Un numero complesso può essere descritto come $z=x+iy$, con $z,y \in \mathbb{R}$: esiste una relazione tra la successione di un numero complesso e la successione della sua parte reale e immaginaria, esposta dal seguente teorema:

\begin{theorem}
    La successione $(z_n)_n$, posto $z_n=x_n+i \cdot y_n$ converge a $\gamma = \alpha + i \beta$ \textbf{se e solo se}
    \[\lim_{n \to +\infty} x_n = \alpha \hspace{1em} \text{ e } \hspace{1em} \lim_{n \to + \infty} y_n = \beta\]
\end{theorem}

\vspace{2em}
\noindent
\normalfont \normalsize
\textsc{Dimostrazione 1}: Dalla definizione di modulo, si ha che
\[\vert x_n - \gamma \vert = \sqrt{(x_n-\alpha)^2+(y_n-\beta)^2}\]
Da ciò appare evidente che
\[\vert x_n - \alpha \vert \leq \vert z_n - \gamma \vert\]
\[\vert y_n - \beta  \vert \leq \vert z_n - \gamma \vert\]

\vspace{2em}
\noindent
\normalfont \normalsize
\textsc{Dimostrazione 2}: ... continua ... Dalla definizione di modulo, si ha che
\[\vert x_n - \gamma \vert = \sqrt{(x_n-\alpha)^2+(y_n-\beta)^2}\]
Da ciò appare evidente che
\[\vert x_n - \alpha \vert \leq \vert z_n - \gamma \vert\]
\[\vert y_n - \beta  \vert \leq \vert z_n - \gamma \vert\]

\vspace{1em}
\noindent
\textbf{Osservazione}: Dal momento che le serie sono particolari successioni, tali risultati si applicano in modo identico. Per cui una serie a termini complessi
\[\sum_{n=0}^{+\infty} z_n\]
converge \textbf{se e solo se} convergono le serie
\[\sum_{n=0}^{+\infty} \text{Re}(z_n) \hspace{1em} \text{e} \hspace{1em} \sum_{n=0}^{+\infty} \text{Im}(z_n)\]
e si ha
\[\sum_{n=0}^{+\infty} z_n = \sum_{n=0}^{+\infty} \text{Re}(z_n) + i \sum_{n=0}^{+\infty} \text{Im}(z_n)\]

\newpage
\begin{center}
    10 Ottobre 2022
\end{center}
Le serie numeriche sono delle coppie di successioni: una è la successione dei termini generali, l'altra è la successione delle somme parziali.\\
Se una successione è convergente, allora il suo termine generale è infinitesimo. Una serie può essere sempre pensata come un integrale generalizzato.\\
Le serie a termini (reali) positivi sono le serie più facili da studiare, in forza del teorema dell'aut-aut.\\
Il criterio di convergenza più importante è il criterio dell'ordine di infinitesimo, a cui si aggiunge il criterio del rapporto e il criterio della radice $n$-esima.\\
Tuttavia, se una serie non è a termini (reali) positivi, si può associare ad essa la serie dei suoi moduli, che è a termini positivi, quindi più facile da studiare. Una serie si dice assolutamente convergente se la serie dei suoi moduli è convergente.\\
Serie non assolutamente convergenti vengono chiamate serie \textbf{semplicemente convergenti}.

\vspace{1em}
\subsection{Serie semplicemente convergenti}

\vspace{1em}
\noindent
\subsubsection{Criterio di Leibniz per le serie a termini alterni}
Si consideri $(a_n)_n$ una successione a termini reali, con $a_n \in \mathbb{R}$, tale che
\begin{itemize}
    \item $a_n > 0, \hspace{0.5em} \forall n \in \mathbb{N}$
    \item $a_{n+1} \leq a_n, \hspace{0.5em} \forall n \in \mathbb{N}$
    \item il termine $a_n$ deve essere infinitesimo:
    \[\lim_{n \to +\infty} a_n = 0\]
\end{itemize}
Allora la serie costruita come
\[\sum_{n=0}^{+\infty} (-1)^n \cdot a_n\]
converge.

% Formattazione per la dimostrazione, etc.
\vspace{2em}
\noindent
\normalfont \normalsize
\textsc{Dimostrazione}: Si consideri la ridotta $n$-esima $s_n$. Posto $k \in \mathbb{N}$ tale per cui $k \geq 0$, allora studiando la sottosuccessione dei termini pari e quella dei termini dispari, si ha
\begin{enumerate}
    \item $s_{2k+2} = s_{2k} - a_{2k+1} + a_{2k+2} = s_{2k} - (a_{2k+1}-a_{2k+2}) \leq s_{2k}$
    essendo, per ipotesi, $a_{n+1} \leq a_n$, e quindi si ha che $a_{2k+1}-a_{2k+2} \geq 0$.\\
    Per tale ragione, tale sottosuccessione è \textbf{monotona decrescente}.

    \item $s_{2k+3} = s_{2k+1} + a_{2k+2} - a_{2k+3} = s_{2k+1} + (a_{2k+2}-a_{2k+3}) \geq s_{2k+1}$
    essendo, per ipotesi, $a_{n+1} \leq a_n$, e quindi si ha che $a_{2k+2}-a_{2k+3} \geq 0$.\\
    Per tale ragione, tale sottosuccessione è \textbf{monotona crescente}.
\end{enumerate}
È noto, per ipotesi, che
\[s_{2k+1} - s_{2k} = (-1)^{2k+1} \cdot a_{2k+1} = -a_{2k+1} \leq 0 \hspace{1em} \text{ e quindi } s_{2k+1} \leq s_{2k}, \hspace{1em} \forall k \geq 0\]
ciò significa che, per ogni $n$, la ridotta pari è maggiore della ridotta dispari, rimbalzando progressivamente attorno al limite delle due sottosuccessioni.\\
Dalle disuguaglianze di cui sopra si ha che
\begin{align}
    s_{2k} \geq s_{2k+1} \geq s_1 = a_0-a_1 & \forall k > 0\\
    s_{2k+1} \leq s_{2k} \leq s_2 = a_0-a_1+a_2 & \forall k > 0
\end{align}
Essendo le due sottosuccessioni, decrescente e crescente, rispettivamente limitata dal basso e dall'alto, esiste per entrambe un limite finito:
\[\lim_{k \to +\infty} s_{2k} = \beta \hspace{1em} \text{e} \hspace{1em} \lim_{k \to + \infty} s_{sk+1} = \alpha\]
e, per il teorema del confronto dei limiti, si ha che $\alpha \leq \beta$.\\
Essendo il termine $a_n$ infinitesimo, si ha che
\[0 = \lim_{n \to +\infty} a_n = \lim_{k \to +\infty} a_{2k+1} = \lim_{k \to +\infty} s_{2k} - s_{sk+1} = \alpha - \beta = 0\]
Dal momento che le sottosuccessioni sono complementari, la serie di partenza converge.

\vspace{1em}
\noindent
\textbf{Osservazione}: Inoltre, detta $s$ la somma della serie, si ha che
\[\forall n \hspace{1em} \left \vert s_n-s \right \vert \leq a_{n+1}\]
secondo la cosiddetta \textbf{formula di approssimazione}. Tale formula funziona in quanto
\begin{itemize}
    \item se $n$ è dispari
    \[s-s_{2k+1} \leq s_{2k+2} - s_{sk+1} = a_{2k+1} + a_{2k+1} - s_{2k+1} = a_{2k+2} = a_{(2k+1)+1}= a_{n+1}\]
    
    \item se $n$ è pari
    \[s_{2k}-s \leq s_{2k} - s_{2k+1} = a_{2k+1} = a_{n+1}\]
\end{itemize}

\vspace{1em}
\noindent
\textbf{Esempio}: Si consideri la serie di Leibniz:
\[\sum_{n=1}^{+\infty} (-1)^n \frac{1}{n} = s\]
Allora, per conoscere la somma della serie con un errore di $\dfrac{1}{10}$, è sufficiente  considerare
\[s_9 = -1 + \frac{1}{2} - \frac{1}{3} + \dots - \frac{1}{9}\]

\vspace{1em}
\noindent
\textbf{Esercizio 1}: Si consideri la serie seguente
\[\sum_{n=1}^{+\infty} (-1)^n \cdot \frac{\log_{10}(n)}{n}\]
Si controlli se sono verificate le condizioni seguenti
\begin{itemize}
    \item Si ha che
    \[\frac{\log_{10}(n)}{n} > 0 \hspace{1em} \forall n \geq 2\]

    \item Si ha che
    \[\lim_{n \to +\infty} a_n = \lim_{n \to +\infty} \frac{\log_{10}(n)}{n} = 0\]

    \item La successione
    \[\frac{\log_{10}(n)}{n}\]
    è decrescente?
\end{itemize}
Per verificare l'ultimo punto, si considera la funzione
\[f(x) = \frac{\log_{10}(x)}{x}\]
e se ne calcola la derivata, da cui
\[f'(x) = \frac{\frac{1}{x \cdot \log(10)} \cdot x - \log_{10}(x) \cdot 1}{x^2}\]
Se ne studia il segno, che dipende solamente dal numeratore, da cui
\[\frac{1}{x \cdot \log(10)} \cdot x - \log_{10}(x) \cdot 1 > 0 \hspace{1em} \rightarrow \hspace{1em} \log_{10}(x) < \frac{1}{\log(10)} \hspace{1em} \rightarrow \hspace{1em} x < 10^{\log(10)}\]
Per cui per $x > 10^{\log(10)}$, la funzione è decrescente. Per tale ragione, la serie
\[\sum_{n=3}^{+\infty} \frac{\log_{10}(n)}{n}\]
converge ad $s$. Tuttavia, non è possibile applicare la formula di approssimazione, in quanto le condizioni di Leibniz non sono soddisfatte per tutti gli $n$.

\vspace{1em}
\noindent
\textbf{Esercizio 2}: Si consideri la serie seguente
\[\sum_{n=0}^{+\infty} \frac{\sin \left(\frac{\pi}{3} \cdot (1+3n)\right)}{1+3n}\]
che, in prima approssimazione, sembra non essere assolutamente convergente, in quanto il suo comportamento asintotico risulta essere simile a quello della serie armonica.\\
Per verificare se essa sia convergente semplicemente, si verifica se essa soddisfa le tre condizioni di Leibniz; riscrivendo il termine generale si ha
\[\sin \left(\frac{\pi}{3} + 3 n\right) = -\sin\left(\frac{\pi}{3}\right) = (-1)^n \cdot \frac{\sqrt{3}}{2}\]
Ecco, quindi, che la serie può essere riscritta come
\[\sum_{n=0}^{+\infty} (-1)^n \cdot \underbrace{\frac{\frac{\sqrt{3}}{2}}{1+3n}}_{a_n}\]

\vspace{1em}
\noindent
\textbf{Osservazione}: Si presti particolare attenzione che, in questo ultimo caso, è stato fondamentale riscrivere il termine generale, mettendo in evidenza il fattore $(-1)^n$, in quanto per verificare le $3$ ipotesi del criterio di Leibniz, bisogna studiare il termine
\[\frac{\frac{\sqrt{3}}{2}}{1+3n}\]
che risulta essere
\begin{enumerate}
    \item a termini positivi
    \item infinitesimo
    \item decrescente
\end{enumerate}
Se ne evince che la serie di partenza è convergente per Leibniz.

\vspace{1em}
\noindent
\textbf{Esercizio}: Si consideri la seguente serie, posto $\alpha \in \mathbb{R}$
\[\sum_{n=1}^{+\infty} \frac{\alpha^n + (-5)^n}{2^n} \cdot \sin \left(\pi + \frac{1}{n}\right)\]
Il seno può essere riscritto come
\[\sin \left(\pi + \frac{1}{n}\right) = - \sin \left(\frac{1}{n}\right)\]
Pertanto si ottiene
\[\sum_{n=1}^{+\infty} - \frac{\alpha^n+(-5)^n}{5^n} \cdot \sin \left(\frac{1}{n}\right)\]
Tuttavia, si può osservare immediatamente che se $\left \vert \alpha \right \vert > 5$, la serie non converge in quanto il termine generale non è infinitesimo. Se $\alpha=-5$, si ottiene il termine generale
\[- \frac{2 \cdot (-1)^n \cdot 5^n}{5^n} \cdot \sin \left(\frac{1}{n}\right) = -2 \cdot (-1)^n \cdot \sin \left(\frac{1}{n} \right)\]
in cui il termine
\[\sin \left(\frac{1}{n} \right)\]
soddisfa le $3$ condizioni di Leibniz, quindi la serie di partenza converge.\\
Nel caso in cui $\alpha=5$, si ottiene
\[- \frac{5^n + (-1)^n \cdot 5^n}{5^n} \cdot \sin\]
... continua ...

\vspace{1em}
\noindent
Nel caso in cui $\vert \alpha \vert < 5$, spezzando la frazione si ottiene
\[\left(\frac{\alpha}{5}\right)^n \cdot \left(-\sin \left( \frac{1}{n}\right)\right) + (-1)^n \cdot \sin \left(\frac{1}{n}\right)\]
in cui la prima converge, se confrontata con la geometrica, e anche la seconda converge per Leibniz.

\vspace{1em}
\noindent
\subsection{Successione di Couchy}
Di seguito si espone la definizione di \textbf{successione di Couchy}:

% Tabella per le definizione di concetti, etc...
\vspace{1em}
\rowcolors{1}{black!5}{black!5}
\setlength{\tabcolsep}{14pt}
\renewcommand{\arraystretch}{2}
\noindent
\begin{tabularx}{\textwidth}{@{}|P|@{}}
    \hline
    {\textbf{SUCCESSIONE DI COUCHY}}\\
    \parbox{\linewidth}{Sia $(z_n)_n$ una successione in $\mathbb{C}$, si dirà che $(z_n)_n$ è una successione di Couchy se
    \[\forall \epsilon > 0, \exists n_\epsilon \in \mathbb{N \hspace{0.5em}} \vert \hspace{0.5em} \forall n \geq n_\epsilon, \forall p \in \mathbb{N} \rightarrow \left \vert z_{n+p} - z_n \right \vert < \epsilon\]
    \vspace{-3mm}}\\
    \hline
\end{tabularx}

\vspace{2em}
\noindent
\textbf{Osservazione}: Si osservi che se esiste finito
\[\lim_{n \to +\infty} z_n = l\]
allora la successione è di Couchy.

% Formattazione per la dimostrazione, etc.
\vspace{2em}
\noindent
\normalfont \normalsize
\textsc{Dimostrazione}: Fissato $\epsilon > 0, \exists n \epsilon \in \mathbb{N}$ tale che $\forall n \geq n_\epsilon$, si ha che
\[\left \vert z_n - l \right \vert < \frac{\epsilon}{2}\]
Allora, $\forall n \geq n_\epsilon$, $\forall p \in \mathbb{N}$, si ottiene
\[\left \vert z_{n+p} - z_n \right \vert \leq \left \vert z_{n+p} - l\right \vert + \left \vert l - z_n\right \vert < \epsilon\]

\vspace{1em}
\noindent
\begin{theorem}
    Ogni successione di Couchy in $\mathbb{C}$ (o in $\mathbb{R}$) è convergente.
\end{theorem}

\vspace{2em}
\noindent
\textbf{Osservazione}: Per le serie $(s_n)_n$, si ha che
\[\left \vert \sum_{k=0}^{n+p} a_k - \sum_{k=0}^n a_k \right \vert < \epsilon \hspace{1em} \rightarrow \hspace{1em} \left \vert \sum_{k=n+1}^{n+p} a_k \right \vert\]

\vspace{1em}
\noindent
\subsection{Criterio di Cauchy per la convergenza di una serie}
Una serie
\[\sum a_n\]
converge \textbf{se e solo se} $\forall \epsilon > 0, \exists n_\epsilon \in \mathbb{N}$ tale che $\forall n \geq n_\epsilon$ e $\forall p \in \mathbb{N}$ vale
\[\left \vert \sum_{k=n+1}^{n+p} a_k \right \vert < \epsilon\]

\newpage
\section{Successioni e serie di funzioni}
Di seguito si introduce l'importante tema delle successioni e delle serie di funzioni, in cui.

\vspace{1em}
\subsection{Successioni di funzioni}
Se, per esempio, si introduce una successione di funzioni come la seguente
\[f_n(x) = x^n\]
si ottiene
\begin{enumerate}
    \item $f_0(x)=1$
    \item $f_1(x)=x$
    \item $f_2(x)=x^2$
    \item $f_3(x)=x^3$
\end{enumerate}
o ancora, nel caso di
\[f_n(x) = \cos(xn)\]
si ottiene
\begin{enumerate}
    \item $f_0(x)=\cos(0)=1$
    \item $f_1(x)=\cos(x)$
    \item $f_2(x)=\cos(2x)$
    \item $f_3(x)=\cos(3x)$
\end{enumerate}

\vspace{1em}
\noindent
\subsubsection{Limite di una successione di funzioni}
Sia $f_n : E \longmapsto \mathbb{R}(\mathbb{C})$ e $f : E \longmapsto \mathbb{R}$. Si dice che la successione $\left(f_n\right)_n$ converge puntualmente a $f$ se, $\forall x \in E$
\[\lim_{n \to +\infty} f_n(x) = f(x)\]

\vspace{1em}
\noindent
\textbf{Esempio 1}: Si consideri la successione di funzioni
\[f_n(x) = \cos(nx)\]
allora tale successione ammette limite $0$ se $x=0$, non esiste altrimenti.

\vspace{1em}
\noindent
\textbf{Esempio 2}: Si consideri la successione di funzioni
\[f_n(x) = \frac{1}{x^2+n}\]
tale per cui, $\forall x \in \mathbb{R}$
\[\lim_{n \to +\infty} \frac{1}{x^2+n} = 0\]

\vspace{1em}
\noindent
\textbf{Esempio 3}: Si consideri la successione di funzioni
\[f_n(x) = \frac{n x}{nx^2 + 1}\]
allora si ha che
\[\lim_{n \to +\infty} f_n(x) = \left\{
    \rowcolors{1}{white}{white}
    \setlength{\tabcolsep}{4pt}
    \renewcommand{\arraystretch}{1.2}
    \begin{array}{lll}
        0           & \text{se} & x = 0\\
        \dfrac{1}{x} & \text{se} & x \neq 0
    \end{array}
\right.\]

\newpage
\noindent
\begin{center}
    11 Ottobre 2022
\end{center}
Il criterio di Leibniz è un criterio fondamentale per capire la convergenza semplice di una serie a termini alternativamente positivi e negativi. Dopodiché sono state introdotte le successioni Cauchy e il criterio di Cauchy consente di capire se esiste un limite, senza conoscere il valore del limite, che risulta fondamentale, anche per le dimostrazioni del seguito:

% Formattazione per la dimostrazione, etc.
\vspace{2em}
\noindent
\normalfont \normalsize
\textsc{Dimostrazione 1}: Si consideri la serie
\[\sum_{n=1}^{+\infty} a_n\]
allora si dirà che la serie converge assolutamente se vale la serie dei moduli è convergente, ossia la serie
\[\sum_{n=1}^{+\infty} \left \vert a_n \right \vert\]
è convergente.\\
Se una serie
\[\sum_{n=1}^{+\infty} a_n\]
converge assolutamente, allora la serie dei moduli
\[\sum_{n=1}^{+\infty} \left \vert a_n \right \vert\]
è di Cauchy, ossia, secondo la definizione del criterio Cauchy, si ha che
\[\forall \epsilon > 0, \exists n_\epsilon \in \mathbb{N} \vert \forall n \geq n_\epsilon, \forall p \in \mathbb{N} \hspace{1em} \text{ si ha } \left \vert \sum_{k=n+1}^{n+p} \left \vert a_k \right \vert \right \vert= \sum_{k=n+1}^{n+p} \left \vert a_k \right \vert < \epsilon\]
Per dimostrare che la serie dei moduli è di Cauchy, si sfrutta la disuguaglianza triangolare (ossia il modulo della somma è minore della somma dei moduli), per cui
\[\left \vert \sum_{k=n+1}^{n+p} a_k \right \vert \leq \sum_{k=n+1}^{n+p} \vert a_k \vert \leq ... continua ...\]

\vspace{1em}
\noindent
Si dimostri, tramite il criterio di Cauchy, che la serie armonica è divergente a $+\infty$.

% Formattazione per la dimostrazione, etc.
\vspace{2em}
\noindent
\normalfont \normalsize
\textsc{Dimostrazione 2}: Considerando la ridotta $n$-esima della serie armonica, si ha
\[s_n = \sum_{k=1}^{n} \frac{1}{k}\]
che, per come è stata costruita, è positiva e crescente, per cui, per il teorema di esistenza del limite delle successioni monotone, si può affermare che
\[\exists \lim_{n \to + \infty} s_n\] 
finito o infinito. Si procede, ora, per assurdo, dimostrando che
\[\forall \epsilon > 0, \exists n_\epsilon \in \mathbb{N} : \forall n \geq n_\epsilon \text{ e } \forall p \in \mathbb{N}, \sum_{k=n+1}^{n+p} a_k < \epsilon\]
Allora, posto $n \geq n_\epsilon$, si ha
\[\sum_{k=n+1}^{n+p} a_k = \frac{1}{n+1} + \frac{1}{n+2} + \dots + \frac{1}{n+p} < \epsilon\]
siccome il numero di addendi sommati è pari a $p$, in quanto $(n+p)-(n+1)+1=p$. Fissato $p=n$, si ha che
\[\underbrace{\frac{1}{n+1}}_{>\frac{1}{2n}} + \underbrace{\frac{1}{n+2}}_{>\frac{1}{2n}} + \dots + \frac{1}{2n} < \epsilon\]
ma essendo $n$ addendi, si ottiene che
\[\frac{1}{2n} \cdot n = \frac{1}{2} < \epsilon\]
Ma se si sceglie arbitrariamente $\epsilon>0$, come $\epsilon=\frac{1}{10}$, si incorre in un assurdo.

\vspace{1em}
\noindent
Si consideri la successione di funzioni
\[f_n : [0,1[ \hspace{1em} \text{ definita come } f_n(x) = x^n\]
Allora $\forall x \in [0,1[$, si ha che
\[\lim_{n \to +\infty} x^n = 0\]
e, per la definizione di limite, si ha
\[\forall \epsilon > 0, \exists n_\epsilon \in \mathbb{N} : \forall n \geq n_\epsilon, \vert x^n \vert < \epsilon\]
Per determinare $n_\epsilon$ tale per cui $\forall n \geq n_\epsilon, \vert x^n \vert < \epsilon$, posto $\epsilon=e^{-10}$, si ha
\[x^n < \epsilon \rightarrow e^{n \cdot \log(x)} < \epsilon \rightarrow e^{n \cdot \log(x)} < e^{\log(\epsilon)}\]
Siccome $\log(x) < 0$, si ottiene
\[n > \frac{\log(\epsilon)}{\log(x)}\]
È facile capire che
\[\lim_{x \to 1^-} \frac{\log(\epsilon)}{\log(x)} = +\infty\]
ovvero, $n$ dipende fortemente da $x$: più $x$ tende a $1$ da sinistra, più $n$ deve essere grande al fine di soddisfare il limite di partenza:
\[\lim_{n \to +\infty} x^n = 0\]
Questo perché $0$ è limite puntuale e non uniforme per la successione $f_n$.

% Tabella per le definizione di concetti, etc...
\vspace{1em}
\rowcolors{1}{black!5}{black!5}
\setlength{\tabcolsep}{14pt}
\renewcommand{\arraystretch}{2}
\noindent
\begin{tabularx}{\textwidth}{@{}|P|@{}}
    \hline
    {\textbf{LIMITE UNIFORME}}\\
    \parbox{\linewidth}{Sia $(f_n)_n$ una successione di funzioni $f$, con
    \[f_n : E \longmapsto \mathbb{R}\]
    Si dirà che $f$ è limite uniforme della successione
    \[\lim_{n \to +\infty} f_n = f\]
    uniforme se
    \[\forall \epsilon > 0, \exists n_\epsilon \in \mathbb{N} \text{ tale che } \forall n \geq n_\epsilon, \forall x \in E, \left \vert f_n(x) - f(x)\right \vert < \epsilon\]
    \vspace{-1mm}}\\
    \hline
\end{tabularx}

\vspace{1em}
\noindent
\textbf{Osservazione}: Nel caso di \textbf{limite puntuale}, invece, si ha che
\[\bf{\forall x \in E}, \forall \epsilon > 0, \exists n_{\epsilon, x} \in \mathbb{N} \text{ tale che } \forall n \geq n_\epsilon, \left \vert f_n(x) - f(x) \right \vert < \epsilon\]
in cui è fondamentale capire la forte dipendenza da $x$ in questo caso, cosa che invece non accade nel caso di un limite uniforme, in cui $n_\epsilon$ si mantiene costante indipendentemente dalla scelta di $x$.

\vspace{1em}
\noindent
\textbf{Osservazione}: Data la successione di funzioni
\[f_n(x) = \frac{1}{n+x^2}\]
posto $x=0$, il valore massimo è $\frac{1}{n}$, per cui la successione converge uniformemente. Ciò significa che, fissato $\epsilon$, il grafico di tutte le funzioni in dipendenza da $n$ sono tutte contenute al di sotto del grafico.

\vspace{1em}
\noindent
\textbf{Esercizio 1}: Si consideri la successione di funzioni:
\[f_n(x) = \frac{n}{x^2+n}\]
allora
\[\lim_{n \to +\infty} f_n(x) = 1\]
Naturalmente si ha convergenza puntuale, ma non uniforme. Infatti, se fosse uniforme, fissato $\epsilon=\dfrac{1}{100}$ dovrebbe esistere $n_epsilon \in \mathbb{N}$ tale che $\forall n \geq n_\epsilon, \forall x \in \mathbb{R}$
\[\left \vert \frac{n}{x^2+n} - 1\right \vert < \frac{1}{100}\]
Per dimostrare che ciò non è possibile $\forall x \in \mathbb{R}$, si sviluppa, ottenendo
\[\left \vert \frac{n-x^2-n}{x^2+n} \right \vert = \frac{x^2}{x^2 + n}\]
basta scegliere $x=\sqrt{n}$, per cui
\[\frac{n}{n+n}=\frac{1}{2} < \frac{1}{100}\]
che, ovviamente è falso.

\vspace{1em}
\noindent
\textbf{Esercizio 2}: Si consideri la successione di funzioni
\[f_n(x) = \frac{nx}{nx^2+1}\]
allora
\[\lim_{n \to +\infty} f_n(x) = \frac{1}{x}\]
che è una convergenza puntuale, ma non uniforme, in quanto, fissato $\epsilon=\dfrac{1}{100}$ dovrebbe esistere $n_epsilon \in \mathbb{N}$ tale che $\forall n \geq n_\epsilon, \forall x \in \mathbb{R}$
\[\left \vert \frac{nx}{nx^2+1} - \frac{1}{x}\right \vert < \frac{1}{100}\]
e sviluppando si ottiene
\[\frac{1}{x \cdot (nx^2+1)} < \frac{1}{100}\]
in cui basta scegliere $x=\dfrac{1}{\sqrt{n}}$, ottenendo
\[\frac{\sqrt{n}}{2} < \frac{1}{100}\]
che, ovviamente, non è vero $\forall n \in \mathbb{N}$.

\vspace{1em}
\noindent
\textbf{Osservazione}: Si osservi che se il limite di una successione di funzioni è discontinuo, allora la successione non converge uniformemente.

\vspace{1em}
\noindent
\subsection{Teorema di inversione di due limiti}
Si consideri il seguente \textbf{teorema di inversione di due limiti}:

\begin{theorem}
    Sia $f(n)_n$ una successione di funzioni
    \[f_n : E \longmapsto \mathbb{R}\]
    tale che $(f_n)_n$ \textbf{converge uniformemente} a
    \[f : E \longmapsto \mathbb{R}\]
    con $x_0$ punto di accumulazione per $E, \forall n$, tale per cui
    \[\exists \lim_{x \to x_0} f_n(x) = l_n\]
    Allora
    \[\exists \lim_{n \to +\infty} l_n = 0, \hspace{1em} \exists \lim_{x \to x_0} f(x) \hspace{1em} \exists \lim_{x \to x_0} f(x) = l\]
    si può affermare che
    \[\lim_{n \to +\infty} \left(\lim_{x \to x_0} f_n(x)\right) = \lim_{x \to x_0} \left(\lim_{n \to +\infty} f_n(x)\right)\]
\end{theorem}

\vspace{1em}
\noindent
\textbf{Osservazione}: Si osservi che quanto esposto in precedenza vale solamente per successioni di funzioni con convergenza uniforme, non puntuale. Infatti
\[\lim_{n \to +\infty} \left(\lim_{x \to 1} x^n\right) = 1 \neq 0 = \lim_{x \to 1} \left(\lim_{n \to +\infty} f_n(x)\right)\]

% Formattazione per la dimostrazione, etc.
\vspace{2em}
\noindent
\normalfont \normalsize
\textsc{Dimostrazione 1}: Per la dimostrazione si considera il criterio di Cauchy, fondamentale per dimostrare l'esistenza del limite
\[\lim_{n \to +\infty} l_n\]
senza conoscerlo. Bisogna dimostrare che la successione $(l_n)_n$ è di Cuachy, ossia che $\forall \epsilon > 0$, $\exists n_\epsilon \in \mathbb{N}$ tale che $\forall n \geq n_\epsilon$ e $\forall p \in \mathbb{N}$
\[\left \vert l_{n+p} - l_n \right \vert < \epsilon\]
In particolare, $\forall x \in E$ si può affermare che
\[\left \vert l_{n+p} - l_n\right \vert = \left \vert l_{n+p} - l_n - f_{n+1}(x) + f_{n+p}(x) - f_(n+1)(x) - f_{n+p}(x)\right \vert\]
Sfruttando la disuguaglianza triangolare, si ha che
\[\left \vert l_{n+p} - l_n - f_{n+1}(x) + f_{n+p}(x) - f_(n+1)(x) - f_{n+p}(x)\right \vert \leq \left \vert l_{n+p} - f_{n+p}(x)\right \vert + \left \vert f_{n+1}(x) - f_n(x) \right \vert + \left \vert f_n(z) - l_n\right \vert\]
Siccome $(f_n)_n$ è uniformemente convergente, quindi è una successione di Cauchy. Ciò significa che
\[\exists n_\epsilon \in \mathbb{N} \vert \forall n \geq n_\epsilon, \forall p \in \mathbb{N} \text{ e } \bf{\forall x \in E} \rightarrow \left \vert f_{n+1}(x) - f_n(x)\right \vert < \frac{\epsilon}{3}\]
Fissato $\hat n \geq n_\epsilon$ e un qualsiasi $q \in \mathbb{N}$, è noto che
\[\lim_{x \to x_0} f_{\hat n}(x) = l_{\hat n} \text{ e } \lim_{x \to x_0} f_{\hat n + p}(x) = l_{\hat n + p}\]
Allora, dalla definizione di limite si ha che
\[\exists \delta_{\hat n + p} > 0, \delta_{\hat n} > 0 \vert \forall x \in E, x \neq x_0, \left \vert x - x_0 \right \vert < \delta_{\hat n + p} \text{ e } \forall x \in E, x \neq x_0, \vert x - x_0 \vert < \delta_{\hat n} \hspace{1em} \text{ si ha } \left \vert f_{\hat n}(x) - l_{\hat n} \right \vert < \frac{\epsilon}{3} \hspace{1em} \text{e} \hspace{1em} \left \vert f_{\hat n + p}(x) - l_{\hat n + p}\right \vert < \frac{\epsilon}{3}\]
Ma ciò consente di affermare che, preso il $\delta$ più piccolo di entrambi $\delta_{\hat n + p}$ e $\delta_{\hat n}$:
\[\left \vert l_{\hat n + \hat p} - l_{\hat n} \right \vert \leq \left \vert l_{\hat n + \hat p} - f_{\hat n + \hat p}\right \vert + \left \vert f_{\hat n + \hat p}(x) - f_{\hat n}(x)\right \vert + \left \vert f_{\hat n}(x) - l_{\hat n}\right \vert < \epsilon\]
Ciò, quindi, consente di affermare che
\[\exists \lim_{n \to +\infty} l_n = l \rightarrow \left \vert l_{n+p} - l_n \right \vert < \epsilon\]

% Formattazione per la dimostrazione, etc.
\vspace{2em}
\noindent
\normalfont \normalsize
\textsc{Dimostrazione 2}: Ripetendo la dimostrazione per
\[\lim_{x \to x_0} f(x) = l\]
si ha che
\[\left \vert f(x) - l \right \vert \leq \left \vert f(x) - l + l_n + f_n(x) - f_n(x) \right \vert \leq \left \vert - \left(f_n(n) - f(x) \right) \right \vert + \left \vert f_n(x) - l_n \right \vert + \left \vert l_n - l \right \vert < \epsilon\]
Infatti, è noto che
\[\lim_{n \to +\infty} l_n = l\]
Pertanto esiste $n^1_\epsilon$ tale che $\forall n \geq n^1_\epsilon$ si ha
\[\left \vert l_n - l \right \vert < \frac{1}{3} \epsilon\]
Inoltre, poiché
\[\lim_{n \to +\infty} f_n = f\]
uniforme, esiste $n^2_\epsilon \in \mathbb{N}$ tale che $\forall n \geq n^2_\epsilon$, si ha
\[\left \vert f_n(x) - f(x) \right \vert < \frac{\epsilon}{3}, \hspace{1em} \forall x \in E\]
Fissato, quindi, $\hat n \geq \max \{n^1_\epsilon,n^2_\epsilon\}$. Per questo $\hat n$, siccome
\[\lim_{x \to x_0} f_{\hat n}(x) = l_{\hat n}\]
si ha che $\exists \delta_\epsilon > 0$ tale che $\forall x \in E, x \neq x_0, \left \vert x - x_0 \right \vert < \delta_\epsilon$
\[\left \vert f_{\hat n}(x) - l_{\hat n} \right \vert < \frac{\epsilon}{3}\]


\vspace{1em}
\noindent
\textbf{Ricapitolando}: Bisogna dimostrare che
\[\left \vert f(x) - l \right \vert \leq \left \vert f(x) - f_n(x) \right \vert\]
... continua ...

\vspace{1em}
\begin{corollary}
    Si osservi che se
    \[f_n : E \longmapsto \mathbb{R}\]
    è continua $\forall n$ e 
    \[\lim_{n\to +\infty} f_n = f\]
    uniforme. Allora $f$ è continua.
\end{corollary}

% Formattazione per la dimostrazione, etc.
\vspace{2em}
\noindent
\normalfont \normalsize
\textsc{Dimostrazione}: Ciò è immediatamente evidente in quanto
\[\lim_{x \to x_0} f(x) = \lim_{x \to x_0} \left(\lim_{n \to +\infty} f_n(x)\right) = \lim_{n \to +\infty} \left(\lim_{x \to x_0} f_n(x)\right) = \lim_{n \to +\infty} f_n(x_0) = f(x_0)\]

\vspace{1em}
\noindent
\textbf{Osservazione}: Si osservi che
\[\lim_{n \to +\infty} f_n(x) = f(x)\]
uniforme, lo è anche puntuale.

\vspace{1em}
\noindent
\subsection{Teorema di integrabilità}
Sia $I \subseteq \mathbb{R}$ un \textbf{intervallo compatto} (ovvero con misura finita) e sia 
\[f_n : I \longmapsto \mathbb{R}\]
integrabile $\forall n$; sia, inoltre
\[\lim_{n \to +\infty} f_n = f\]
uniforme, con
\[f : I \longmapsto \mathbb{R}\]
allora $f$ è integrabile e si ha che
\[\int_I \lim_{n \to +\infty} f_n(x) \dif x = \lim_{n \to +\infty} \int_I f_n(x) \dif x\]

% Formattazione per la dimostrazione, etc.
\vspace{2em}
\noindent
\normalfont \normalsize
\textsc{Dimostrazione}: Parlando di integrale di Riemann, si dimostra che
\[\left \vert \int_I f_n(x) \dif x - \int_I f(x) \dif x \right \vert \leq \int_I \left \vert f_n(x) - f(x) \right \vert \dif x < \epsilon \cdot m(I)\]
in quanto $\left \vert f_n(x) - f(x) \right \vert < \epsilon, \forall x$ se $n \geq n_epsilon$ per la convergenza uniforme.

\vspace{1em}
\noindent
\textbf{Dimostrazione}: Si consideri la seguente successione di funzioni
\[f_n : [0,1] \longmapsto \mathbb{R}\]
in cui
\[f_n(x) = \left\{
    \begin{array}{lll}
        0 & \text{ se } & x \in \{0\} \cup \left]\frac{1}{n+1},1\right]\\
        n & \text{ se } & x \in \left]0,\frac{1}{n}\right]
    \end{array}
\right.\]
in cui appare evidente come
\[\int_[0,1] f(x) \dif x = 1, \forall n\]
e, ovviamente,
\[\lim_{n \to +\infty} \int_{[0,1]} f_n(x) = 1\]
mentre
\[\int_{[0,1]} \left(\lim_{n \to +\infty} f_n(x) \right) \dif x = \int_{[0,1]} 0 \dif x = 0\]


\end{document}